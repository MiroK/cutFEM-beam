\documentclass[a4paper,10pt]{article}
\pdfoptionpdfminorversion=5
\usepackage[utf8]{inputenc}
\usepackage{amsmath}
\usepackage{amssymb}
\usepackage{amsfonts}
\usepackage{multicol}
\usepackage{xcolor,colortbl}
\usepackage{datetime}
\usepackage{xfrac}

\newcommand{\mc}[2]{\multicolumn{#1}{c}{#2}}
\definecolor{Gray}{gray}{0.85}
\newcolumntype{a}{>{\columncolor{Gray}}c}
\newcolumntype{b}{>{\columncolor{white}}c}
\newcolumntype{R}{>{\columncolor{lime}}c}
\newcolumntype{B}{>{\columncolor{pink}}c}
\usepackage{diagbox}

\usepackage[numbers]{natbib}

\usepackage[utf8]{inputenc}
\usepackage{array,multirow,graphicx}
\usepackage{graphics}
\usepackage{color}
\usepackage{ifpdf}
\ifpdf
\DeclareGraphicsRule{*}{mps}{*}{}
\fi
\usepackage[utf8]{inputenc}

\hoffset = 0pt
\voffset = 0pt
\oddsidemargin = 0pt
\headheight = 15pt
\footskip = 30pt
\topmargin = 0pt
\marginparsep = 5pt
\headsep =25pt
\marginparwidth = 54pt
\marginparpush = 7pt
\textheight = 621pt %621/
\textwidth = 500pt

% Default fixed font does not support bold face
\DeclareFixedFont{\ttb}{T1}{txtt}{bx}{n}{12} % for bold
\DeclareFixedFont{\ttm}{T1}{txtt}{m}{n}{12}  % for normal

\newcommand{\I}{\ensuremath{\mathbb{I}}}
\newcommand{\R}{\ensuremath{\mathbb{R}}}

\newcommand{\norm}[1]{\ensuremath{\left\|#1\right\|}}
\newcommand{\seminorm}[1]{\ensuremath{\left|#1\right|}}
\newcommand{\dual}[2]{\ensuremath{\langle#1, #2 \rangle}}

\newcommand{\inner}[2]{\ensuremath{\left(#1, #2\right)}}
\newcommand{\Inner}[2]{\ensuremath{\left(\left(#1, #2\right)\right)}}

\newcommand{\deriv}[2]{\ensuremath{\frac{\mathrm{d}#1}{\mathrm{d}#2}}}
\newcommand{\meas}[1]{\ensuremath{\,\mathrm{d}#1}}
\newcommand{\Div}[1] {\ensuremath{\text{div}#1}}
\newcommand{\Grad}[1]{\ensuremath{\text{grad}#1}}
\newcommand{\Curl}[1]{\ensuremath{\text{curl}#1}}
\newcommand{\jump}[1]{\ensuremath{[\![#1]\!]} }

\newcommand{\citeneed}{\ensuremath{^\text{\textcolor{blue}{cite?}}}}

\usepackage{mathtools}
\newcommand{\eqdef}{\mathrel{\mathop=}:} 

\newcommand{\mm}{\ensuremath{\mathbf{m}}}
\newcommand{\Vp}{\ensuremath{V_{\mathcal{P}}}}
\newcommand{\Vb}{\ensuremath{V_{\mathcal{B}}}}
\newcommand{\Ep}{\ensuremath{E_{\mathcal{P}}}}
\newcommand{\Eb}{\ensuremath{E_{\mathcal{B}}}}

%matrices
\newcommand{\Ap}{\ensuremath{\mathbb{A}_{\mathcal{P}}}}
\newcommand{\Ab}{\ensuremath{\mathbb{A}_{\mathcal{B}}}}
\newcommand{\Bp}{\ensuremath{\mathbb{B}_{\mathcal{P}}}}
\newcommand{\Bb}{\ensuremath{\mathbb{B}_{\mathcal{B}}}}
\newcommand{\Amat}{\ensuremath{\mathbb{A}}}
\newcommand{\Bmat}{\ensuremath{\mathbb{B}}}
\newcommand{\Bmatt}{\ensuremath{\mathbb{B}^{\text{T}}}}
\newcommand{\Mmat}{\ensuremath{\mathbb{M}}}
\newcommand{\Nmat}{\ensuremath{\mathbb{N}}}
%components
\newcommand{\Apij}[2]{\ensuremath{\left(\mathbb{A}_{\mathcal{P}}\right)_{#1, #2}}}
\newcommand{\Abij}[2]{\ensuremath{\left(\mathbb{A}_{\mathcal{B}}\right)_{#1, #2}}}
\newcommand{\Bpij}[2]{\ensuremath{\left(\mathbb{B}_{\mathcal{P}}\right)_{#1, #2}}}
\newcommand{\Bbij}[2]{\ensuremath{\left(\mathbb{B}_{\mathcal{B}}\right)_{#1, #2}}}

\newcommand{\up}{\ensuremath{u_{\mathcal{P}}}}
\newcommand{\ub}{\ensuremath{u_{\mathcal{B}}}}
\newcommand{\vp}{\ensuremath{v_{\mathcal{P}}}}
\newcommand{\vb}{\ensuremath{v_{\mathcal{B}}}}
\newcommand{\TODO}[1]{\textcolor{red}{\MakeUppercase{#1}}}
\newcommand{\ASK}[1]{\textcolor{blue}{#1}}

\newcommand{\W}[1]{\ensuremath{w\!\left[#1\right]\!}}
\newcommand{\E}[1]{\ensuremath{\epsilon \!\left[#1\right]\!}}
\newcommand{\T}[1]{\ensuremath{\sigma \! \left[#1\right]\!}}

\newcommand{\Tr}[1]{\ensuremath{\text{tr}#1}}
\newcommand*{\defeq}{\mathrel{\vcenter{\baselineskip0.5ex \lineskiplimit0pt
                     \hbox{\scriptsize.}\hbox{\scriptsize.}}}%
                     =}

\usepackage{lipsum}
\usepackage{authblk}
\usepackage{fancyhdr}

\usepackage{chngcntr}
\counterwithin{table}{subsection}

\DeclareMathOperator{\spn}{span}

% DEBUGGING
\usepackage{lineno}
\linenumbers
%\usepackage{setspace}
%\doublespacing
%
%\pagestyle{fancy}
%
\renewenvironment{abstract}{%
\hfill\begin{minipage}{0.95\textwidth}
\rule{\textwidth}{1pt}}
{\par\noindent\rule{\textwidth}{1pt}\end{minipage}}
%
\makeatletter
\renewcommand\@maketitle{%
\hfill
\begin{minipage}{0.95\textwidth}
\vskip 2em
\let\footnote\thanks 
{\LARGE \@title \par }
\vskip 1.5em
{\large \@author \par}
\end{minipage}
\vskip 1em \par
}
\makeatother
%
\begin{document}
%
%title and author details
\title{\begin{center}
        Note on plate-beam-like problems
       \end{center}}
\author[1]{MK}
%\author[2]{Name2}
%\affil[1]{Address of author} 
%\affil[2]{Address of second author}
%
\maketitle
%
\begin{abstract}
  This report is a summary of observations made on the properties of saddle
  point systems that arise when a physical process in a domain
  $\mathcal{P}\subset\R^2$ is coupled to another physical process in a domain
  $\mathcal{B}\subset\mathcal{P}$ with topological dimension equal to one and
  the coupling constraint does not involve a differential operator. We discuss
  two cases (i) the governing equations of both processes are given by the
  Laplace operator, (ii) the governing equations of both processes are given by
  the biharmonic operator.
\end{abstract}

\section{Introduction}
  We consider $\mathcal{P}=\left[-1, 1\right]^2$ and refer to this domain as
  plate. To define $\mathcal{B}$ we let
  $\vec{\chi}, \vec{\chi}:\left[-1, 1\right]\mapsto\mathcal{P}$, denote some
  invertible mapping from the bi-unit interval onto the plate. Then
  $\mathcal{B}=\left\{\vec{x}\in\R^2, \vec{x}=\vec{\chi}(s), s\in\left[-1,
  1\right]\right\}$. For simplicity we shall have
  $\vec{\chi}(s)=\frac{\vec{A}}{2}(1-s) +\frac{\vec{B}}{2}(1+s)$ for some
  distinct points $\vec{A}, \vec{B}\in\partial\mathcal{P}$. We then refer to
  the set $\mathcal{B}$ as beam. Further we let $V_{\mathcal{P}}, V_{\mathcal{B}}$
  be the spaces of functions that map plate and the beam respectively to real
  numbers. We postpone discussion of regularity properties of functions in these
  two spaces for later sections.

  The cases considered in this report are instances of the following
  problem: Find $\up\in\Vp, \ub\in\Vb$ that minimize a functional
  \begin{equation}
    \label{eq:energy}
    \begin{aligned}
      E(\vp, \vb) &= \Ep(\vp, \vp) + \Eb(\vb, \vb) - \int_{\mathcal{P}}f \vp \meas{x}\meas{y}, \\
      \text{subject to the constraint}&\\
      T(\vp) &= \vb\text{ on }\mathcal{B},
    \end{aligned}
  \end{equation}
  where $\Ep, \Ep: \Vp\times\Vp\mapsto\R$, is a bilinear form that describes
  energy of the physical process on the plate and similarly the bilinear form
  $\Eb, \Eb:\Vb\times\Vb\mapsto\R$, describes energy of the beam process. The
  last term in the expression is a potential energy due to forcing $f$. Finally
  we denote as $T, T:\Vp\mapsto\Vb$, the trace operator.
  
  The problem (\ref{eq:energy}) can be also recast into an unconstrained
  optimization setting, where we search for extrema of the Lagrangian
  \begin{equation}
    \label{eq:lagrangian}
    L(\vp, \vb, q) = \Ep(\vp, \vp) + \Eb(\vb, \vb) - \int_{\mathcal{P}}f \vb \meas{x}\meas{y} -
    \int_{\mathcal{B}}(T(\vp) - \vb)q \meas{x}.
  \end{equation}
  Here, the function $q$, $q\in Q$, is an unknown Lagrange multiplier
  and the function space $Q$ has functions that map the beam to real scalars.
  Regularity of functions in $Q$ is discussed later.
  An extreme point of the Lagrangian (\ref{eq:lagrangian}) is given as a
  solution of the problem: Find $\up\in\Vp, \ub\in\Vb, p\in Q$ such that
  for all $\vp\in\Vp, \vb\in\Vb, q\in Q$ it holds that
 \begin{equation}
    \label{eq:system}
    \begin{aligned}
      \Ep(\up, \vp) - \int_{\mathcal{B}}T(\vp)p\meas{x} &=
      \int_{\mathcal{P}}f\vp\meas{x}\meas{y}, \\
      \Eb(\ub, \vb) + \int_{\mathcal{B}}\vb p\meas{x} &= 0, \\
      \int_{\mathcal{B}}(T(\up)-\ub)q\meas{x}\meas{y} &= 0.
    \end{aligned}
  \end{equation}

  \subsection{Abstract saddle point problem}
  To put problem (\ref{eq:system}) into the framework of abstract saddle point
  problems we define a space $V=\Vp \times \Vb$ and bilinear forms
  $a, a:V\times V\mapsto\R$ such that for $V\ni v=(\vp, \vb)$ we let
  $a(v, v)=\Ep(\vp, \vp) + \Eb(\vb, \vb)$ and $b, b:V\times Q\mapsto\R$ such
  that $b(v, q)= - \int_{\mathcal{B}}T(\vp)q\meas{x} +
  \int_{\mathcal{B}}\vb q\meas{x}$. Finally we shall define a
  linear form over $V\times Q$ such that $L((v, q))=\int_\mathcal{P}f \vp$. With
  these definitions (\ref{eq:system}) can be written as: Find $(u, p)\in
  V\times Q$ such that for all $(v, q)\in V\times Q$ it holds that 
  \begin{equation}
    \label{eq:abstract_saddle}
    a(u, v) + b(u, q) + b(v, p) = L((v, q)). 
  \end{equation}
  
  Before discussing existence and uniqueness of the solution of
  (\ref{eq:abstract_saddle}) we equip the spaces $V, Q, V\times Q$ with norms.
  If $\norm{\cdot}_{\Vp}$ and $\norm{\cdot}_{\Vb}$ are norms on the spaces $\Vp$
  and $\Vb$ we define the norm on $V$ as 
  \[
    \norm{v}_V=\sqrt{\norm{\vp}^2_{\Vp} + \norm{\vb}^2_{\Vb}}.
  \] Further, we let $\norm{\cdot}_{Q}$
  denote the norm on space $Q$ and finally define the norm on space $V\times Q$
  as 
  \[
    \norm{(v, q)}_{V\times Q}=\sqrt{\norm{v}^2_V +\norm{q}^2_Q}.
  \]
  Problem (\ref{eq:abstract_saddle}) has a unique solution iff there exists
  positove a $\gamma$ defined as
  \begin{equation}
    \label{eq:babuska}
    \gamma=\inf_{(u, p)\neq 0}\sup_{(v, q) \neq 0}
    \frac{a(u, v) + b(u, q) + b(v, p)}
    {\norm{(u, p)}_{V\times Q}  \norm{(v, q)}_{V\times Q}}.
  \end{equation}
  Condition (\ref{eq:babuska}) is due to Babuska(\cite{babuska_error,
  babuska_lag}). Babuska condition is equivalent(\cite{equiv}) to two Brezzi
  conditions(\cite{brezzi}) which
  state that (\ref{eq:abstract_saddle}) has unique solution if (i) the
  bilinear form $a$ is coercive on $Z=\left\{z\in V; b(z, q)=0\,\forall\,q\in Q\right\}$:
  \begin{equation}
    \label{eq:brezzi_coer}
    \inf_{u\in Z}\sup_{v\in Z} \frac{a(u, v)}{\norm{u}_V\norm{v}_V} = \alpha > 0
  \end{equation}
  and (ii) the bilinear form $b$ satisfies the inf sup condition
  \begin{equation}
    \label{eq:brezzi_infsup}
    \inf_{q\in Q}\sup_{u \in V} \frac{b(u, q)}{\norm{u}_V\norm{q}_Q}
    = \beta > 0.
  \end{equation}
  
  The Brezzi conditions can be easily understood in the light of the following
  two step algorithm(\cite{scott}) which solves the problem (\ref{eq:abstract_saddle}). First
  we observe that the solution $(u, p)\in V\times Q$ is clearly such that
  $u\in Z$. Moreover, using test functions in $Z\subset V$ we have that $u$
  is a solution to
  \[
    a(u, v) = \int_{\mathcal{P}} f v\meas{x}\meas{y}\,\forall v\in Z.
  \]
  Once $u$ is found, the multiplier $p$ is obtained from
  \[
    b(v, p) = \int_{\mathcal{P}} f v\meas{x}\meas{y} - a(u, v)\,\forall v\in V.
  \]
  The Brezzi conditions are necessary conditions for existence of unique
  $u\in Z$ and $p\in Q$ as solutions to the respected subproblems. Further
  we note that the Brezzi inf-sup condition prohibits existence functions
  $0\neq p^{\star}\in Q$ for which $b(v, p^{\star})=0\,\forall v\in V$.
  Such functions are referred to as the spurious modes and their existence
  implies that the solution to the saddle point is not unique. Indeed, given
  $(u, p)$ a solution to (\ref{eq:abstract_saddle}) a tuple $(u, p+p^{\star})$
  is also a solution of the saddle point problem. Finally we note that if
  the spurious modes exist we have $\beta=0$.

  Each of the conditions (\ref{eq:babuska}, \ref{eq:brezzi_coer},
  \ref{eq:brezzi_infsup}) can be associated with
  an eigenvalue problem. It is not difficult to verify that the constant $\gamma$
  from the Babuska condition satisfies $\gamma=\left|\lambda_{\text{min}}\right|$, where
  $\lambda_\text{min}$ is the smallest eigenvalue of the eigenproblem: Find
  $\lambda\in\R, (u, p) \in V\times Q$ such that
  \begin{equation}
    a(u, v) + b(u, q) + b(v, p) = \lambda\left(m(u, v) + n(p, q)\right)\,\forall
    (v, q) \in V\times Q.
  \end{equation}
  \TODO{exaplain $m, n$}. Further, \cite{qin} gives that the Brezzi coercivity
  constant satisfies $\beta=\sqrt{\lambda_{\text{min}}}$, where 
  $\lambda_\text{min}$ is the smallest eigenvalue of the eigenproblem: Find
  $\lambda\in\R, (u, p) \in V\times Q$ such that
  \begin{equation}
    a(u, v) + b(u, q) + b(v, p) = -\lambda n(p, q)\,\forall(v, q) \in V\times Q.
  \end{equation}
  Finally, the inf-sup constant $\beta$ can be computed as the smallest eigen
  vaue of the problem: Find $\lambda\in\R, u, \in Z$ such that
  \begin{equation}
    a(u, v) = \lambda m(u, v)\,\forall v \in Z.
  \end{equation}
  or equivalently (\cite{rognes}) from a problem: Find
  $\lambda\in\R, (u, p) \in V\times Q$ such that
  \begin{equation}
    a(u, v) + b(u, q) + b(v, p) = \lambda m(p, q)\,\forall(v, q) \in V\times Q.
  \end{equation}
  The latter problem is computationally more feasible since it does not involve
  construction of the kernel space $Z$.

  \subsection{Discrete setting}
  To solve (\ref{eq:abstract_saddle}) numerically we choose $\mm\in\mathbb{N}^2, 
r\in\mathbb{N}$ and define $V_\mm, Q_r$ the finite dimensional subspaces
of $V, Q$. Note the subscripts represent dimension of the discrete spaces,
$m=\text{dim}{V}=m_0+m_1, \text{dim}{Q}=r$, which in this study are constructed
by the finite element method(\cite{scott}) and the spectral Galerkin
method(\cite{shen_book}). In both cases
the approximation $(u, p)\in V_{\mm}\times Q_e$ 
to solution of (\ref{eq:abstract_saddle}) satisfies
  \begin{equation}
    \label{eq:abstract_saddle_h}
    a(u, v) + b(u, q) + b(v, p) = L((v, q))\,\forall (v, q) \in V_{\mm}\times Q_r.
  \end{equation}
  For fixed values of $\mm$ and $r$ we have that the problem
  (\ref{eq:abstract_saddle_h}) is well posed if the discrete Babuska condition
  is satisfied:
  \begin{equation}
    \label{eq:babuska_h}
    \gamma_{\mm, r}=\inf_{0\neq(u, p)\in \left(V_{\mm}\times Q_r\right)}
    \sup_{0\neq(u, p)\in \left(V_{\mm}\times Q_r\right)}
    \frac{a(u, v) + b(u, q) + b(v, p)}
    {\norm{(u, p)}_{V\times Q}  \norm{(v, q)}_{V\times Q}}.
  \end{equation}
  or equivalently if the two discrete Brezzi conditions hold:
  \begin{equation}
    \label{eq:brezzi_coer_h}
    \begin{aligned}
      &\inf_{u\in Z_{\mm, r}}
    \sup_{v\in Z_{\mm, r}}
    \frac{a(u,v)}{\norm{u}_V\norm{v}_V} = \alpha_{\mm, r} > 0,\\
    &\quad\text{where}\\
    & Z_{\mm, r} = \left\{z\in V_{\mm}; b(z, q)=0\,\forall\,q\in Q_r\right\}
    \end{aligned}
  \end{equation}
  and 
  \begin{equation}
    \label{eq:brezzi_infsup_h}
    \inf_{q\in Q_r}\sup_{u \in V_\mm} \frac{b(u, q)}{\norm{u}_V\norm{q}_Q}
    = \beta _{\mm, r}> 0.
  \end{equation}
  To compute the approximate solution with the given accuracy the values of
  parameters $\mm, r$ are changed and it is therefore important that
  the above conditions do not seize to hold. We shall therefore require that
  there exists positive constants $\gamma_0$ or $\alpha_0, \beta_0$ such that
  for all parameters $\mm, r$ it holds that $\gamma_0<\gamma_{\mm, r}$ or
  $\alpha_0<\alpha_{\mm, r}, \beta_0<\beta_{\mm, r}$.
 
  At this point it should be noted that even if the condition (\ref{eq:babuska})
  or (\ref{eq:brezzi_coer}, \ref{eq:brezzi_infsup}) hold for the tuple of spaces
  $(V, Q)$ this property does not imply that the discrete conditions
  (\ref{eq:babuska_h}) or (\ref{eq:brezzi_coer_h}, \ref{eq:brezzi_infsup_h})
  hold on the approximate spaces $(V_{\mm}, Q_{r})$. A famous example of
  this phenomenom is the Stokes problem for which the continous conditions are
  verified easily but the approximate spaces for the discrete conditions must
  be constructed with care; the finite element method uses specialized element
  (e.g. \cite{th_mini}) for the velocity and pressure, while spectral Galerkin
  method requires degree of polynomials for the velocity and degree of
  polynomilas for pressure to differ by two (e.g. \cite{canuto}).

  Let as define basis functions $\phi_i, \varphi_i, \psi_i$ such that
  $V_{\mm}=\spn\left\{\phi_i\right\}_{i=0}^{m_0-1} \times
  \spn\left\{\varphi_i\right\}_{i=0}^{m_1-1}$ and $Q_r=\spn\left\{\psi_i\right\}_{i=0}^{r-1}$.
  To simplify the notation we shall let $\phi_i=\varphi_{i-m_0}$ for $i\geq
  m_0$. With an ansatz 
  \[ u=\displaystyle\sum\limits_{i=0}^{m-1}\phi_i U_i
    \quad\text{and}\quad
  p=\displaystyle\sum\limits_{i=0}^{r-1}\psi_i P_i,
  \]
  the system (\ref{eq:abstract_saddle_h}) can be written as: Find
  $\mathbf{U}\in\R^m, \mathbf{P}\in\R^r$ such that
  \begin{equation}
    \label{system}
    \begin{bmatrix}
      \mathbb{A} & \mathbb{B} \\
      \mathbb{B}^{\text{T}} & 0
    \end{bmatrix}
    \,
    \begin{bmatrix}
      \mathbf{U} \\
      \mathbf{P}
    \end{bmatrix}
    =
    \begin{bmatrix}
      \mathbf{b}\\
      0
    \end{bmatrix}.
  \end{equation}
  Here the matrices $\mathbb{A}\in\R^{m\times m}$ and $\mathbb{B}\in\R^{m\times
  r}$ consist of blocks
  \[
    \Amat = 
    \begin{bmatrix}
      \Ap & 0\\
      0 & \Ab
    \end{bmatrix}
    ,\Ap\in\R^{m_0\times m_0}, \Ab\in\R^{m_1\times m_1}
    \quad
    \text{and}
    \quad
    \Bmat = 
    \begin{bmatrix}
      \Bp \\
      \Bb
    \end{bmatrix}
    ,\Bp\in\R^{m_0\times r}, \Bb\in\R^{m_1\times r}
  \]
  whose components are given as
  \[
  \begin{aligned}
    &\Apij{i}{j} = \Ep(\phi_i,
    \phi_j),\quad&\Abij{i}{j}=\Eb(\varphi_i,\varphi_j),\\
    &\Bpij{i}{j} =
    -\int_\mathcal{B}T(\phi_i)\psi_j,\quad&\Bbij{i}{j}=\int_\mathcal{B}\varphi_i\psi_j.\\
  \end{aligned}
  \]
  Vector $\mathbb{b}\in\R^{m}$ has the first $m_0$ components given by
  $b_i=\int_\mathcal{P}f\phi_i$ while the remenaing $m_1$ components is zero.

  \subsection{Eigenvalue problems related to discrete solvability conditions}
  Before discussing eigenvalue problems through which the discrete Babuska
  and Brezzi constants can be computed we shall first motivate their imporance.
  As in the continuous case we do this by considering a two step algorithm
  for solving the linear system (\ref{eq:system}).



  \% matrices \\
  \% conditions and matrices \\
  \% continuous conditions and matrices \\
  \% discrete conditions and matrices \\
  \% Question: where does $\lambda$ live?, in general what we want to do \\
  \% as new section: Poisson\\
  \% remember to show in appendix that the code works for uncoupled. Compare
  with eigen!\\
  \% as new section: Biharmonic (Shen and Eigen are two different problems!)
  
  %\section{Plansu}
  %Problems (i), (ii) are solved by the Galekin method with Fourier basis and
  %the basis due to Shen. In the first case, the Fourier and Shen basis have
  %same bcs. In the latter one the Fourier bases gives conditions on the function
  %value and the value of the second derivative while the Shen basis constrains
  %the function value and the first derivative. \textbf{So it would be nice to 
  %get Shen like basis so that the boundary conditions are the same}.

  %For both (i), (ii) we have numerical experiments with the Fourier basis which
  %investigate preconditionear for the Schur complement. The preconditioner is
  %related to the norm of $\lambda$ in the LBB condition. As part of the
  %preconditioner investigations we can show that the scaling of the preconditioned
  %system is independent of the beam position. What seems to change
  %with the beam position is the condition number of the preconditioned system.
  %A fun question for me is whether stiffer systems mean stiffer plate-beam. 
  %\textbf{Similar experiments must be made with Shen basis}. It is possible that
  %there the scalling of the differential operator might be more difficult to see.
  %Finally at least \textbf{for (i), the above should be possible with FEniCS}.

  %Problems (i), (ii) share the constraint and the trace operator is part of the
  %LBB condition for both. We can construct a matrix that is related to the trace
  %operator with Fourier basis, both Shen basis and perhaps FEniCS. \textbf{We
  %could then ask what the properties are of this matrix for different discretization
  %parameters, $N, h$ and what are their paralles in case of the trace operator.}

  \bibliographystyle{plain}
  \bibliography{plate_beam}
\end{document}
