\documentclass[a4paper,10pt]{article}
\pdfoptionpdfminorversion=5
\usepackage[utf8]{inputenc}
\usepackage{amsmath}
\usepackage{amssymb}
\usepackage{amsfonts}
\usepackage{multicol}
\usepackage{xcolor,colortbl}
\usepackage{datetime}
\usepackage{xfrac}

\newcommand{\mc}[2]{\multicolumn{#1}{c}{#2}}
\definecolor{Gray}{gray}{0.85}
\newcolumntype{a}{>{\columncolor{Gray}}c}
\newcolumntype{b}{>{\columncolor{white}}c}
\newcolumntype{R}{>{\columncolor{lime}}c}
\newcolumntype{B}{>{\columncolor{pink}}c}
\usepackage{diagbox}

\usepackage[numbers]{natbib}

\usepackage[utf8]{inputenc}
\usepackage{array,multirow,graphicx}
\usepackage{graphics}
\usepackage{color}
\usepackage{ifpdf}
\ifpdf
\DeclareGraphicsRule{*}{mps}{*}{}
\fi
\usepackage[utf8]{inputenc}

\hoffset = 0pt
\voffset = 0pt
\oddsidemargin = 0pt
\headheight = 15pt
\footskip = 30pt
\topmargin = 0pt
\marginparsep = 5pt
\headsep =25pt
\marginparwidth = 54pt
\marginparpush = 7pt
\textheight = 621pt %621/
\textwidth = 500pt

% Default fixed font does not support bold face
\DeclareFixedFont{\ttb}{T1}{txtt}{bx}{n}{12} % for bold
\DeclareFixedFont{\ttm}{T1}{txtt}{m}{n}{12}  % for normal

\newcommand{\I}{\ensuremath{\mathbb{I}}}
\newcommand{\R}{\ensuremath{\mathbb{R}}}

\newcommand{\norm}[1]{\ensuremath{\left\|#1\right\|}}
\newcommand{\seminorm}[1]{\ensuremath{\left|#1\right|}}
\newcommand{\dual}[2]{\ensuremath{\langle#1, #2 \rangle}}

\newcommand{\inner}[2]{\ensuremath{\left(#1, #2\right)}}
\newcommand{\Inner}[2]{\ensuremath{\left(\left(#1, #2\right)\right)}}

\newcommand{\deriv}[2]{\ensuremath{\frac{\mathrm{d}#1}{\mathrm{d}#2}}}
\newcommand{\meas}[1]{\ensuremath{\,\mathrm{d}#1}}
\newcommand{\Div}[1] {\ensuremath{\text{div}#1}}
\newcommand{\Grad}[1]{\ensuremath{\text{grad}#1}}
\newcommand{\Curl}[1]{\ensuremath{\text{curl}#1}}
\newcommand{\jump}[1]{\ensuremath{[\![#1]\!]} }

\newcommand{\citeneed}{\ensuremath{^\text{\textcolor{blue}{cite?}}}}

\usepackage{mathtools}
\newcommand{\eqdef}{\mathrel{\mathop=}:} 

\newcommand{\mm}{\ensuremath{\mathbf{m}}}
\newcommand{\Vp}{\ensuremath{V_{\mathcal{P}}}}
\newcommand{\Vb}{\ensuremath{V_{\mathcal{B}}}}
\newcommand{\Ep}{\ensuremath{E_{\mathcal{P}}}}
\newcommand{\Eb}{\ensuremath{E_{\mathcal{B}}}}

%matrices
\newcommand{\Ap}{\ensuremath{\mathbb{A}_{\mathcal{P}}}}
\newcommand{\Ab}{\ensuremath{\mathbb{A}_{\mathcal{B}}}}
\newcommand{\Bp}{\ensuremath{\mathbb{B}_{\mathcal{P}}}}
\newcommand{\Bb}{\ensuremath{\mathbb{B}_{\mathcal{B}}}}
\newcommand{\Amat}{\ensuremath{\mathbb{A}}}
\newcommand{\Bmat}{\ensuremath{\mathbb{B}}}
\newcommand{\Bmatt}{\ensuremath{\mathbb{B}^{\text{T}}}}
\newcommand{\Mmat}{\ensuremath{\mathbb{M}}}
\newcommand{\Nmat}{\ensuremath{\mathbb{N}}}
%components
\newcommand{\Apij}[2]{\ensuremath{\left(\mathbb{A}_{\mathcal{P}}\right)_{#1, #2}}}
\newcommand{\Abij}[2]{\ensuremath{\left(\mathbb{A}_{\mathcal{B}}\right)_{#1, #2}}}
\newcommand{\Bpij}[2]{\ensuremath{\left(\mathbb{B}_{\mathcal{P}}\right)_{#1, #2}}}
\newcommand{\Bbij}[2]{\ensuremath{\left(\mathbb{B}_{\mathcal{B}}\right)_{#1, #2}}}
%vectors
\newcommand{\Uvec}{\ensuremath{\mathbf{U}}}
\newcommand{\bvec}{\ensuremath{\mathbf{b}}}
\newcommand{\Pvec}{\ensuremath{\mathbf{P}}}

\newcommand{\up}{\ensuremath{u_{\mathcal{P}}}}
\newcommand{\ub}{\ensuremath{u_{\mathcal{B}}}}
\newcommand{\vp}{\ensuremath{v_{\mathcal{P}}}}
\newcommand{\vb}{\ensuremath{v_{\mathcal{B}}}}
\newcommand{\TODO}[1]{\textcolor{red}{#1}}
\newcommand{\ASK}[1]{\textcolor{blue}{#1}}

\newcommand{\W}[1]{\ensuremath{w\!\left[#1\right]\!}}
\newcommand{\E}[1]{\ensuremath{\epsilon \!\left[#1\right]\!}}
\newcommand{\T}[1]{\ensuremath{\sigma \! \left[#1\right]\!}}

\newcommand{\Tr}[1]{\ensuremath{\text{tr}#1}}
\newcommand*{\defeq}{\mathrel{\vcenter{\baselineskip0.5ex \lineskiplimit0pt
                     \hbox{\scriptsize.}\hbox{\scriptsize.}}}%
                     =}

\usepackage{lipsum}
\usepackage{authblk}
\usepackage{fancyhdr}
\usepackage[colorlinks=true,
            linkcolor=cyan,
            citecolor=gray]{hyperref}

\usepackage{chngcntr}
\counterwithin{table}{subsection}

\DeclareMathOperator{\spn}{span}

% DEBUGGING
\usepackage{lineno}
\linenumbers
%\usepackage{setspace}
%\doublespacing
%
%\pagestyle{fancy}
%
\renewenvironment{abstract}{%
\hfill\begin{minipage}{0.95\textwidth}
\rule{\textwidth}{1pt}}
{\par\noindent\rule{\textwidth}{1pt}\end{minipage}}
%
\makeatletter
\renewcommand\@maketitle{%
\hfill
\begin{minipage}{0.95\textwidth}
\vskip 2em
\let\footnote\thanks 
{\LARGE \@title \par }
\vskip 1.5em
{\large \@author \par}
\end{minipage}
\vskip 1em \par
}
\makeatother
%
\begin{document}
%
%title and author details
\title{\begin{center}
        Note on plate-beam-like problems
       \end{center}}
\author[1]{MK}
%\author[2]{Name2}
%\affil[1]{Address of author} 
%\affil[2]{Address of second author}
%
\maketitle
%
\begin{abstract}
  This report is a summary of observations made on the properties of saddle
  point systems that arise when a physical process in a domain
  $\mathcal{P}\subset\R^2$ is coupled to another physical process in a domain
  $\mathcal{B}\subset\mathcal{P}$ with topological dimension equal to one and
  the coupling constraint does not involve a differential operator. We discuss
  two cases (i) the governing equations of both processes are given by the
  Laplace operator, (ii) the governing equations of both processes are given by
  the biharmonic operator.
\end{abstract}

\section{Introduction}
  We consider $\mathcal{P}=\left[-1, 1\right]^2$ and refer to this domain as
  plate. To define $\mathcal{B}$ we let
  $\vec{\chi}, \vec{\chi}:\left[-1, 1\right]\mapsto\mathcal{P}$, denote some
  invertible mapping from the bi-unit interval onto the plate. Then
  $\mathcal{B}=\left\{\vec{x}\in\R^2, \vec{x}=\vec{\chi}(s), s\in\left[-1,
  1\right]\right\}$. For simplicity we shall have
  $\vec{\chi}(s)=\frac{\vec{A}}{2}(1-s) +\frac{\vec{B}}{2}(1+s)$ for some
  distinct points $\vec{A}, \vec{B}\in\partial\mathcal{P}$. We then refer to
  the set $\mathcal{B}$ as beam. Further we let $V_{\mathcal{P}}, V_{\mathcal{B}}$
  be the spaces of functions that map plate and the beam respectively to real
  numbers. We postpone discussion of regularity properties of functions in these
  two spaces for later sections.

  The cases considered in this report are instances of the following
  problem: Find $\up\in\Vp, \ub\in\Vb$ that minimize a functional
  \begin{equation}
    \label{eq:energy}
    \begin{aligned}
      E(\vp, \vb) &= \Ep(\vp, \vp) + \Eb(\vb, \vb) - \int_{\mathcal{P}}f \vp \meas{x}\meas{y}, \\
      \text{subject to the constraint}&\\
      T(\vp) &= \vb\text{ on }\mathcal{B},
    \end{aligned}
  \end{equation}
  where $\Ep, \Ep: \Vp\times\Vp\mapsto\R$, is a bilinear form that describes
  energy of the physical process on the plate and similarly the bilinear form
  $\Eb, \Eb:\Vb\times\Vb\mapsto\R$, describes energy of the beam process. The
  last term in the expression is a potential energy due to forcing $f$. Finally
  we denote as $T, T:\Vp\mapsto\Vb$, the trace operator.
  
  The problem (\ref{eq:energy}) can be also recast into an unconstrained
  optimization setting, where we search for extrema of the Lagrangian
  \begin{equation}
    \label{eq:lagrangian}
    L(\vp, \vb, q) = \Ep(\vp, \vp) + \Eb(\vb, \vb) - \int_{\mathcal{P}}f \vb \meas{x}\meas{y} -
    \int_{\mathcal{B}}(T(\vp) - \vb)q \meas{x}.
  \end{equation}
  Here, the function $q$, $q\in Q$, is an unknown Lagrange multiplier
  and the function space $Q$ has functions that map the beam to real scalars.
  Regularity of functions in $Q$ is discussed later.
  An extreme point of the Lagrangian (\ref{eq:lagrangian}) is given as a
  solution of the problem: Find $\up\in\Vp, \ub\in\Vb, p\in Q$ such that
  for all $\vp\in\Vp, \vb\in\Vb, q\in Q$ it holds that
 \begin{equation}
    \label{eq:system}
    \begin{aligned}
      \Ep(\up, \vp) - \int_{\mathcal{B}}T(\vp)p\meas{x} &=
      \int_{\mathcal{P}}f\vp\meas{x}\meas{y}, \\
      \Eb(\ub, \vb) + \int_{\mathcal{B}}\vb p\meas{x} &= 0, \\
      \int_{\mathcal{B}}(T(\up)-\ub)q\meas{x}\meas{y} &= 0.
    \end{aligned}
  \end{equation}

  \subsection{Abstract saddle point problem}
  To put problem (\ref{eq:system}) into the framework of abstract saddle point
  problems we define a space $V=\Vp \times \Vb$ and bilinear forms
  $a, a:V\times V\mapsto\R$ such that for $V\ni v=(\vp, \vb)$ we let
  $a(v, v)=\Ep(\vp, \vp) + \Eb(\vb, \vb)$ and $b, b:V\times Q\mapsto\R$ such
  that $b(v, q)= - \int_{\mathcal{B}}T(\vp)q\meas{x} +
  \int_{\mathcal{B}}\vb q\meas{x}$. Finally we shall define a
  linear form over $V\times Q$ such that $L((v, q))=\int_\mathcal{P}f \vp$. With
  these definitions (\ref{eq:system}) can be written as: Find $(u, p)\in
  V\times Q$ such that for all $(v, q)\in V\times Q$ it holds that 
  \begin{equation}
    \label{eq:abstract_saddle}
    a(u, v) + b(u, q) + b(v, p) = L((v, q)). 
  \end{equation}
  
  Before discussing existence and uniqueness of the solution of
  (\ref{eq:abstract_saddle}) we equip the spaces $V, Q, V\times Q$ with norms.
  If $\norm{\cdot}_{\Vp}$ and $\norm{\cdot}_{\Vb}$ are norms on the spaces $\Vp$
  and $\Vb$ we define the norm on $V$ as 
  \[
    \norm{v}_V=\sqrt{\norm{\vp}^2_{\Vp} + \norm{\vb}^2_{\Vb}}.
  \] Further, we let $\norm{\cdot}_{Q}$
  denote the norm on space $Q$ and finally define the norm on space $V\times Q$
  as 
  \[
    \norm{(v, q)}_{V\times Q}=\sqrt{\norm{v}^2_V +\norm{q}^2_Q}.
  \]
  Problem (\ref{eq:abstract_saddle}) has a unique solution iff there exists
  positove a $\gamma$ defined as
  \begin{equation}
    \label{eq:babuska}
    \gamma=\inf_{(u, p)\neq 0}\sup_{(v, q) \neq 0}
    \frac{a(u, v) + b(u, q) + b(v, p)}
    {\norm{(u, p)}_{V\times Q}  \norm{(v, q)}_{V\times Q}}.
  \end{equation}
  Condition (\ref{eq:babuska}) is due to Babu\v ska(\cite{babuska_error,
  babuska_lag}). Babu\v ska condition is equivalent(\cite{equiv}) to two Brezzi
  conditions(\cite{brezzi}) which
  state that (\ref{eq:abstract_saddle}) has unique solution if (i) the
  bilinear form $a$ is coercive on $Z=\left\{z\in V; b(z, q)=0\,\forall\,q\in Q\right\}$:
  \begin{equation}
    \label{eq:brezzi_coer}
    \inf_{u\in Z}\sup_{v\in Z} \frac{a(u, v)}{\norm{u}_V\norm{v}_V} = \alpha > 0
  \end{equation}
  and (ii) the bilinear form $b$ satisfies the inf sup condition
  \begin{equation}
    \label{eq:brezzi_infsup}
    \inf_{q\in Q}\sup_{u \in V} \frac{b(u, q)}{\norm{u}_V\norm{q}_Q}
    = \beta > 0.
  \end{equation}
  
  The Brezzi conditions can be easily understood in the light of the following
  two step algorithm(\cite{scott}) which solves the problem (\ref{eq:abstract_saddle}). First
  we observe that the solution $(u, p)\in V\times Q$ is clearly such that
  $u\in Z$. Moreover, using test functions in $Z\subset V$ we have that $u$
  is a solution to
  \[
    a(u, v) = \int_{\mathcal{P}} f v\meas{x}\meas{y}\,\forall v\in Z.
  \]
  Once $u$ is found, the multiplier $p$ is obtained from
  \[
    b(v, p) = \int_{\mathcal{P}} f v\meas{x}\meas{y} - a(u, v)\,\forall v\in V.
  \]
  The Brezzi conditions are necessary conditions for existence of unique
  $u\in Z$ and $p\in Q$ as solutions to the respected subproblems. Further
  we note that the Brezzi inf-sup condition prohibits existence functions
  $0\neq p^{\star}\in Q$ for which $b(v, p^{\star})=0\,\forall v\in V$.
  Such functions are referred to as the spurious modes and their existence
  implies that the solution to the saddle point is not unique. Indeed, given
  $(u, p)$ a solution to (\ref{eq:abstract_saddle}) a tuple $(u, p+p^{\star})$
  is also a solution of the saddle point problem. Finally we note that if
  the spurious modes exist we have $\beta=0$.

  Each of the conditions (\ref{eq:babuska}, \ref{eq:brezzi_coer},
  \ref{eq:brezzi_infsup}) can be associated with
  an eigenvalue problem. It is not difficult to verify that the constant $\gamma$
  from the Babu\v ska condition satisfies $\gamma=\left|\lambda_{\text{min}}\right|$, where
  $\lambda_\text{min}$ is the smallest eigenvalue of the eigenproblem: Find
  $\lambda\in\R, (u, p) \in V\times Q$ such that
  \begin{equation}
    a(u, v) + b(u, q) + b(v, p) = \lambda\left(m(u, v) + n(p, q)\right)\,\forall
    (v, q) \in V\times Q.
  \end{equation}
  Here, $m, m:V\times V\mapsto\R$ and $n, n:Q\times Q\mapsto\R$ are symmetric
  bilinear positive definite forms (\ASK{is this correct}) that induce
  respectively the norm on $V$ and $Q$. Further, \cite{qin} gives that the Brezzi
  coercivity constant satisfies $\beta=\sqrt{\lambda_{\text{min}}}$, where 
  $\lambda_\text{min}$ is the smallest eigenvalue of the eigenproblem: Find
  $\lambda\in\R, (u, p) \in V\times Q$ such that
  \begin{equation}
    a(u, v) + b(u, q) + b(v, p) = -\lambda n(p, q)\,\forall(v, q) \in V\times Q.
  \end{equation}
  Finally, the inf-sup constant $\beta$ can be computed as the smallest eigen
  vaue of the problem: Find $\lambda\in\R, u, \in Z$ such that
  \begin{equation}
    a(u, v) = \lambda m(u, v)\,\forall v \in Z.
  \end{equation}
  or equivalently (\cite{rognes}) from a problem: Find
  $\lambda\in\R, (u, p) \in V\times Q$ such that
  \begin{equation}
    a(u, v) + b(u, q) + b(v, p) = \lambda m(p, q)\,\forall(v, q) \in V\times Q.
  \end{equation}
  The latter problem is computationally more feasible since it does not involve
  construction of the kernel space $Z$.

  \subsection{Discrete setting}
  To solve (\ref{eq:abstract_saddle}) numerically we choose $\mm\in\mathbb{N}^2, 
r\in\mathbb{N}$ and define $V_\mm, Q_r$ the finite dimensional subspaces
of $V, Q$. Note the subscripts represent dimension of the discrete spaces,
$m=\text{dim}{V}=m_0+m_1, \text{dim}{Q}=r$, which in this study are constructed
by the finite element method(\cite{scott}) and the spectral Galerkin
method(\cite{shen_book}). In both cases
the approximation $(u, p)\in V_{\mm}\times Q_r$ 
to solution of (\ref{eq:abstract_saddle}) satisfies
  \begin{equation}
    \label{eq:abstract_saddle_h}
    a(u, v) + b(u, q) + b(v, p) = L((v, q))\,\forall (v, q) \in V_{\mm}\times Q_r.
  \end{equation}
  For fixed values of $\mm$ and $r$ we have that the problem
  (\ref{eq:abstract_saddle_h}) is well posed if the discrete Babu\v ska condition
  is satisfied:
  \begin{equation}
    \label{eq:babuska_h}
    \gamma_{\mm, r}=\inf_{(u, p)\in \left(V_{\mm}\times Q_r\right)}
    \sup_{(u, p)\in \left(V_{\mm}\times Q_r\right)}
    \frac{a(u, v) + b(u, q) + b(v, p)}
    {\norm{(u, p)}_{V\times Q}  \norm{(v, q)}_{V\times Q}}.
  \end{equation}
  or equivalently if the two discrete Brezzi conditions hold:
  \begin{equation}
    \label{eq:brezzi_coer_h}
    \begin{aligned}
      &\inf_{u\in Z_{\mm, r}}
    \sup_{v\in Z_{\mm, r}}
    \frac{a(u,v)}{\norm{u}_V\norm{v}_V} = \alpha_{\mm, r} > 0,\\
    &\quad\text{where}\\
    & Z_{\mm, r} = \left\{z\in V_{\mm}; b(z, q)=0\,\forall\,q\in Q_r\right\}
    \end{aligned}
  \end{equation}
  and 
  \begin{equation}
    \label{eq:brezzi_infsup_h}
    \inf_{q\in Q_r}\sup_{u \in V_\mm} \frac{b(u, q)}{\norm{u}_V\norm{q}_Q}
    = \beta _{\mm, r}> 0.
  \end{equation}
  To compute the approximate solution with the given accuracy the values of
  parameters $\mm, r$ are changed and it is therefore important that
  the above conditions do not seize to hold. We shall therefore require that
  there exists positive constants $\gamma_0$ or $\alpha_0, \beta_0$ such that
  for all parameters $\mm, r$ it holds that $\gamma_0<\gamma_{\mm, r}$ or
  $\alpha_0<\alpha_{\mm, r}, \beta_0<\beta_{\mm, r}$.
 
  At this point it should be noted that even if the condition (\ref{eq:babuska})
  or (\ref{eq:brezzi_coer}, \ref{eq:brezzi_infsup}) hold for the tuple of spaces
  $(V, Q)$ this property does not imply that the discrete conditions
  (\ref{eq:babuska_h}) or (\ref{eq:brezzi_coer_h}, \ref{eq:brezzi_infsup_h})
  hold on the approximate spaces $(V_{\mm}, Q_{r})$. A famous example of
  this phenomenom is the Stokes problem for which the continous conditions are
  verified easily but the approximate spaces for the discrete conditions must
  be constructed with care; the finite element method uses specialized element
  (e.g. \cite{th_mini}) for the velocity and pressure, while spectral Galerkin
  method requires degree of polynomials for the velocity and degree of
  polynomilas for pressure to differ by two (e.g. \cite{canuto}).

  Let as define basis functions $\phi_i, \varphi_i, \psi_i$ such that
  $V_{\mm}=\spn\left\{\phi_i\right\}_{i=0}^{m_0-1} \times
  \spn\left\{\varphi_i\right\}_{i=0}^{m_1-1}$ and $Q_r=\spn\left\{\psi_i\right\}_{i=0}^{r-1}$.
  To simplify the notation we shall let $\phi_i=\varphi_{i-m_0}$ for $i\geq
  m_0$. With an ansatz 
  \[ u=\displaystyle\sum\limits_{i=0}^{m-1}\phi_i U_i
    \quad\text{and}\quad
  p=\displaystyle\sum\limits_{i=0}^{r-1}\psi_i P_i,
  \]
  the system (\ref{eq:abstract_saddle_h}) can be written as: Find
  $\mathbf{U}\in\R^m, \mathbf{P}\in\R^r$ such that
  \begin{equation}
    \label{system}
    \begin{bmatrix}
      \mathbb{A} & \mathbb{B} \\
      \mathbb{B}^{\text{T}} & 0
    \end{bmatrix}
    \,
    \begin{bmatrix}
      \mathbf{U} \\
      \mathbf{P}
    \end{bmatrix}
    =
    \begin{bmatrix}
      \mathbf{b}\\
      0
    \end{bmatrix}.
  \end{equation}
  Here the matrices $\mathbb{A}\in\R^{m\times m}$ and $\mathbb{B}\in\R^{m\times
  r}$ consist of blocks
  \[
    \Amat = 
    \begin{bmatrix}
      \Ap & 0\\
      0 & \Ab
    \end{bmatrix}
    ,\Ap\in\R^{m_0\times m_0}, \Ab\in\R^{m_1\times m_1}
    \quad
    \text{and}
    \quad
    \Bmat = 
    \begin{bmatrix}
      \Bp \\
      \Bb
    \end{bmatrix}
    ,\Bp\in\R^{m_0\times r}, \Bb\in\R^{m_1\times r}
  \]
  whose components are given as
  \[
  \begin{aligned}
    &\Apij{i}{j} = \Ep(\phi_i,
    \phi_j),\quad&\Abij{i}{j}=\Eb(\varphi_i,\varphi_j),\\
    &\Bpij{i}{j} =
    -\int_\mathcal{B}T(\phi_i)\psi_j\meas{x},
    \quad&\Bbij{i}{j}=\int_\mathcal{B}\varphi_i\psi_j\meas{x}.\\
  \end{aligned}
  \]
  Vector $\bvec\in\R^{m}$ has the first $m_0$ components given by
  $\bvec_i=\int_\mathcal{P}f\phi_i\meas{x}\meas{y}$ while the remaining $m_1$
  components are zero. Finally for the future reference we define symmetric
  positive definite matrices $\Mmat\in\R^{m\times m}, \Nmat\in\R^{n\times n}$
  with components
  \[
    \left(\Mmat\right)_{i, j} = m(\phi_i, \phi_j)
    \quad\text{and}\quad
    \left(\Nmat\right)_{i, j} = n(\psi_i, \psi_j).
  \]

  \subsection{Eigenvalue problems related to discrete solvability conditions}
  Before discussing eigenvalue problems through which the discrete Babu\v ska
  and Brezzi constants can be computed we shall first motivate their importance.
  As in the continuous case we do this by considering a two step algorithm
  for solving the linear system (\ref{system}). Suppose the $\Amat$ is
  invertible and vector $\Pvec$ is known, then from the first equation in the
  system we get that 
  \begin{equation}
    \label{eq:sys_step1}
    \Uvec = \Amat^{-1}\left(\bvec - \Bmat\Pvec\right),
  \end{equation}
  which can be substituted into the second equation of (\ref{system})
  yielding
  \begin{equation}
    \label{eq:sys_step2}
    \Bmatt\Amat^{-1}\left(\Bmat\Pvec-\bvec\right) = 0.
  \end{equation}
  The latter linear system is solvable iff $\Bmat\Pvec=0\Leftrightarrow\Pvec=0$,
  which is guaranteed by the Brezzi infsup condition (\ref{eq:brezzi_infsup_h}).
  Coming back to the first linear system, its well-posedness requires
  $\Amat\Uvec=0\Leftrightarrow\Uvec=0$ but the condition does not need to hold
  on entire $\R^m$. By the second equation of (\ref{system}) we have
  $\Bmatt\Uvec=0$ and thus we only need $\Amat\Uvec=0\Leftrightarrow\Uvec=0$ for
  $\Bmatt\Uvec=0$. Recalling definition of the space $Z_{\mm, r}$ and matrix
  $\Bmat$ we see that the kernel can be identified with $Z_{m,
  r}=\left\{\Uvec\in\R^m, \Bmatt\Uvec=0\right\}$ allowing us to write the
  invertibility condition as $\Amat\Uvec=0\Leftrightarrow\Uvec=0$ for
  $\Uvec\in Z_{m, r}$. But this is clearly equivalent to asking for the matrix
  $\Amat$ to be positive definite on $Z_{m, r}$ which is the statement of the
  discrete Brezzi coercivity condition (\ref{eq:brezzi_infsup_h}).

  Similar to their continuous counterparts, constants from discrete Babu\v ska
  and Brezzi conditions can be computed from suitable matrix eigenvalue
  problems. The Babuska constant $\gamma_{\mm, r}$ is the smallest in magnitude
  eigenvalue of the following problem: Find $\lambda\in\R, \Uvec\in\R^m,
  \Pvec\in\R^r$ that solve
  \begin{equation}
    \label{eq:babuska_sys}
    \begin{bmatrix}
      \mathbb{A} & \mathbb{B} \\
      \mathbb{B}^{\text{T}} & 0
    \end{bmatrix}
    \,
    \begin{bmatrix}
      \mathbf{U} \\
      \mathbf{P}
    \end{bmatrix}
    =
    \lambda
    \begin{bmatrix}
      \Mmat & 0\\
      0 & \Nmat
    \end{bmatrix}.
  \end{equation}
  Several observations can be made about the eigenvectors of the problem. (i)
  Consider the solution of the form $\lambda, \Uvec, 0$. We get that the
  triplet is admissible iff $\lambda, \Uvec$ solve
  \[
    \begin{aligned}
      &\Amat\Uvec = \lambda\Mmat\Uvec,\\
      &\Bmat\Uvec = 0.
    \end{aligned}
  \]
  Consequently if $\lambda=0$ is in the spectrum, the first subproblem
  (\ref{eq:sys_step1}) is not solvable. (ii) Consider the solution of the form
  $\lambda, 0, \Pvec$. It follows that the triplet is admissible iff
  \[
    \begin{aligned}
      &\Bmat\Pvec = 0,\\
      &\lambda = 0. 
    \end{aligned}
  \]
  The spurious modes are therefore associated with $\lambda=0$ meaning that
  the subproblem (\ref{eq:sys_step2}) is not well-posed.
  (iii) Assume $\Amat$ is invertible (note that this equivalent to problem (i)
  having only nonzero eigenvalues) and set $\Mmat=\Amat$ transforming
  (\ref{eq:babuska_sys}) into: Find $\lambda\in\R, \Uvec\in\R^m, \Pvec\in\R^r$
  that solve
  \[
    \begin{bmatrix}
      \mathbb{A} & \mathbb{B} \\
      \mathbb{B}^{\text{T}} & 0
    \end{bmatrix}
    \,
    \begin{bmatrix}
      \mathbf{U} \\
      \mathbf{P}
    \end{bmatrix}
    =
    \lambda
    \begin{bmatrix}
      \Amat & 0\\
      0 & \Nmat
    \end{bmatrix}.
  \]
  The above can be also stated as: Find $\lambda\in\R, \Pvec\in\R^r$ that solves
  \begin{equation}
    \label{eq:S_eig}
    \Bmatt\Amat^{-1}\Bmat\Pvec=\lambda\left(1-\lambda\right)\Nmat\Pvec.
  \end{equation}
  We observe that the well-posedness of (\ref{system}) then hinges only on the
  properties of the Schur complement $S=\Bmatt\Amat^{-1}\Bmat$. This is an
  advantage from the computational point of view since the matrix involved in 
  the eigenvalue problem in only $r\times r$. Further, for well-posedness we
  must have $\lambda\notin\left\{0, 1\right\}$. Finally, observe that due to
  block structure of $\Amat, \Bmat$, the left hand side of (\ref{eq:S_eig})
  can be written as
  \[
    \Bp\Ap^{-1}\Bp^{\text{T}}\Pvec + \Bb\Ab^{-1}\Bb^{\text{T}}\Pvec.
  \]
  Broadly speaking problem (\ref{eq:S_eig}) can be viewed as combination
  of two problems:
  \begin{equation}
    \label{eq:Sp_eig}
    \Bp\Ap^{-1}\Bp^{\text{T}}\Pvec = \Nmat_\mathcal{P}\Pvec,
  \end{equation}
  and
  \begin{equation}
    \label{eq:Sb_eig}
    \Bb\Ab^{-1}\Bb^{\text{T}}\Pvec = \Nmat_\mathcal{B}\Pvec.
  \end{equation}

  In order to compute the discrete Brezzi constants $\alpha_{\mm, r}$ and
  $\beta_{\mm, r}$ we consider the following eigenvalue problems: Find
  $\lambda, \Uvec\in\R^m, \Pvec\in\R^n$ such that
  \begin{equation}
    \label{eq:brezzi_coer_sys}
    \begin{bmatrix}
      \mathbb{A} & \mathbb{B} \\
      \mathbb{B}^{\text{T}} & 0
    \end{bmatrix}
    \,
    \begin{bmatrix}
      \mathbf{U} \\
      \mathbf{P}
    \end{bmatrix}
    =
    \lambda
    \begin{bmatrix}
      \Mmat & 0\\
      0 & 0
    \end{bmatrix},
  \end{equation}
  Find $\lambda, \Uvec\in\R^m, \Pvec\in\R^n$ such that
  \begin{equation}
    \label{eq:brezzi_infsup_sys}
    \begin{bmatrix}
      \Mmat & \mathbb{B} \\
      \mathbb{B}^{\text{T}} & 0
    \end{bmatrix}
    \,
    \begin{bmatrix}
      \mathbf{U} \\
      \mathbf{P}
    \end{bmatrix}
    =
    -\lambda
    \begin{bmatrix}
      0 & 0\\
      0 & \Nmat
    \end{bmatrix}.
  \end{equation}
  Using result from \cite{rognes} we have that the triplet $\lambda, \Uvec,
  \Pvec$ solving (\ref{eq:brezzi_coer_sys}) is such that $\lambda, \Uvec$ solves
  \[
    \begin{aligned}
      &\Amat\Uvec = \lambda\Mmat\Uvec,\\
      &\Bmat\Uvec = 0.
    \end{aligned}
  \]
  For the latter eigenvalue problem we clearly have 
  $\left|\lambda_{\text{min}}\right|=\alpha_{\mm, r}$ and by the equivalence
  the discrete Brezzi coercivity constant can be computed as the smallest
  eigenvalue of (\ref{eq:brezzi_infsup_sys}). Finally \cite{qin} gives that
  discrete Brezzi infsup constant $\beta_{\mm, r}$ can be computed as
  $\beta_{\mm, r}=\sqrt{\lambda_\text{min}}$ where the eigenvalues of
  (\ref{eq:brezzi_infsup_sys}) are to be considered.

  \section{Poisson problem}
  In this section we consider a special case of (\ref{eq:abstract_saddle}) where
  the bilinear forms $\Ep, \Eb$ take the form
  \[\Ep(\vp,\vp)=\int_\mathcal{P}\Grad{\vp}\cdot\Grad{\vp}\meas{x}\meas{y}
    \quad\text{and}\quad
  \Eb(\vb,\vb)=\int_\mathcal{B}\deriv{\vb}{\vec{x}}\cdot\deriv{\vb}{\vec{x}}\meas{x}.
  \]
  It is more practical to use the mapping $\chi$ and instead of using functions
  that map the beam $\mathcal{B}$ to scalars consider function defined over the
bi-unit interval. We then define the spaces for the plate unknown,
beam unknown and the Lagrange multiplier as 
$\Vp=H^1_0\left(\mathcal{P}\right), \Vb=H^1_0\left(\left(-1, 1\right)\right),
Q=L^2\left(\left(-1, 1\right)\right)$ and the bilinear forms as
\[
  \begin{aligned}
  &a(u, v) = \Ep(\up, \vp) + \Eb(\ub, \vb) = \int_\mathcal{P}\Grad{\vp}\cdot\Grad{\vp}\meas{x}\meas{y}
  + \frac{2}{L}\int_{-1}^{1}\deriv{\ub}{s}\deriv{\vb}{s}\meas{s},\\
  &b(u, q) = -\frac{L}{2}\int_{-1}^{1}T(\up)q\meas{s} +
  \frac{L}{2}\int_{-1}^{1}\ub q\meas{s},\\
  \end{aligned}
\]
where $L$ denotes the length of the beam. We note that by Poincare inequality,
the bilinear form $a$ defines a norm on $\Vp$.\\
\TODO{
\% Need existence and uniqueness in this case for (\ref{eq:system})\\
\% Need discrete conditions, whatever they say (maybe compatibility) needs to
be reflected in numerical experiments\\
}
We shall now consider problem (\ref{eq:system}, with the bilinear forms $a, b$
defined above, discretized by the finite element method and Galerkin methods
with either the eigenfunctions of Laplacian or polynomials due to \cite{shen_leg}
employed as the trial/test functions. We are interested in the stability of each
disretization. To this end we shall study eigenvalue problems (\ref{eq:S_eig},
\ref{eq:Sp_eig}, \ref{eq:Sb_eig} and \ref{eq:babuska_sys}). To make sure that
our findings do not depend on the particular position of the beam, we shall 
consider multiple configurations. The configurations invstigated in this report
are pictured in Figure \ref{fig:beam_positions}.
\begin{figure}
  \begin{center}
    \includegraphics[width=0.75\textwidth]{img/beam_positions.pdf}
  \caption{System configurations of the plate-system problem. The blue line
  denotes the beam.}
  \label{fig:beam_positions}
  \end{center}
\end{figure}

Let $\left\{\phi_j\right\}_{j=0}^{m-1}$ be either the set of eigenfunctions or
polynomials. For simplicity we construct the finite dimensional subspaces only
from this set. Namely, we let $Q_m=\spn\left\{\phi_j(s)\right\}_{j=0}^{m-1}$,
and $V_\mm=\spn\left\{\phi_i(x)\phi_j(y)\right\}_{i, j=0}^{m-1,m-1}\times Q_m$.
We thus have $\text{dim}{Q_r}=m$ and $\text{dim}{V_\mm}=m^2+m$. For the space
$V_\mm, Q_m$ and a given position of the beam, we now consider eigenvalue
problems (\ref{eq:S_eig}, \ref{eq:Sp_eig}, \ref{eq:Sb_eig} and
\ref{eq:babuska_sys}).

\subsection{Galerkin method with eigenfunctions}
The results for beam positions 0, 7, 12, 21 are summarized in Figures
\ref{fig:pois_eigen_1}, \ref{fig:pois_eigen_2}, \ref{fig:pois_eigen_3} and
\ref{fig:pois_eigen_4}. Figure \ref{fig:pois_eigen_1} shows that the smallest
eigenvalues are bounded from below if the matrix $\Nmat$ is due to
$H^{-1/2}$ or $H^{-1}$ norms. Observe that the $L^2$ norm, which 
translates to identy matrix yields eigenvalues that decrease for all $m$.
\begin{figure}
  \begin{center}
    \includegraphics[width=0.75\textwidth]{img/eigen_(i).pdf}
    \caption{Eigen.Smallest eigenvalues of problem (\ref{eq:S_eig}) with $\Nmat$ as
    a matrix of $L^2(H^0), H^{-1/2}$ and $H^{-1}$ norms.}
  \label{fig:pois_eigen_1}
  \end{center}
\end{figure}
Figures \ref{fig:pois_eigen_2}, \ref{fig:pois_eigen_3} show the smallest
eigenvalues of problems (\ref{eq:Sp_eig}), and (\ref{eq:Sb_eig}) respectively.
Note that $\Nmat_\mathcal{P}$ due to $H^{-1/2}$ or $H^{-1}$ norm is
needed to obtain a lower bound on the eigenvalues of the plate problem. On the
other hand only matrix $\Nmat_\mathcal{B}$ due to $H^{-1}$ gives bounded
eigenvalues for the beam problem. This is understandable, since in this case
$\Bp$ is and idenity matrix and so the result says \textit{$\Ab^{-1}$ scales as
$\Ab^{-1}$}.
\begin{figure}
  \begin{center}
    \includegraphics[width=0.75\textwidth]{img/eigen_(ii).pdf}
    \caption{Eigen.Smallest eigenvalues of problem (\ref{eq:Sp_eig}) with
      $\Nmat_\mathcal{P}$ as
    a matrix of $L^2(H^0), H^{-\tfrac{1}{2}}$ and $H^{-1}$ norms.}
  \label{fig:pois_eigen_2}
  \end{center}
\end{figure}
\begin{figure}
  \begin{center}
    \includegraphics[width=0.75\textwidth]{img/eigen_(iii).pdf}
    \caption{Eigen.Smallest eigenvalues of problem (\ref{eq:Sb_eig}) with
      $\Nmat_\mathcal{B}$ as
    a matrix of $L^2(H^0), H^{-\tfrac{1}{2}}$ and $H^{-1}$ norms.}
  \label{fig:pois_eigen_3}
  \end{center}
\end{figure}
Finally figure $\ref{fig:pois_eigen_4}$ shows the dependence of discrete 
Babuska constant for the discretiazation method.
In agreement with our previous findings the lower bound is obtained only if the
matrix $\Nmat$ is due to $H^{-1}$ or $H^{-1/2}$ norms.
\begin{figure}
  \begin{center}
    \includegraphics[width=0.75\textwidth]{img/eigen_(iv).pdf}
    \caption{Eigen.Smallest eigenvalues of problem (\ref{eq:babuska}) with $\Nmat$ as
    a matrix of $L^2(H^0), H^{-\tfrac{1}{2}}$ and $H^{-1}$ norms and
  $\Mmat=\Amat$}
  \label{fig:pois_eigen_4}
  \end{center}
\end{figure}

\subsection{Galerkin method with Legendre polynomials}
The results of eigenvalue problems with matrices due to Galerkin method 
with basis from combinations of Legendre polynomials are shown in Figures 
\ref{fig:pois_shen_1}, \ref{fig:pois_shen_2}, \ref{fig:pois_shen_3}, 
\ref{fig:pois_shen_4}. We note that in this basis $\Amat_1$, the stiffness matrix 
of one dimensional Laplacian, is an identity matrix. Therefore the curves
in figures \ref{fig:pois_shen_1}, \ref{fig:pois_shen_2}, \ref{fig:pois_shen_3}
corresponding to $H^{1}$ norm or no preconditionting(identy matrix) collide.
The first three figures show that there exists no lower bound on the eigenvalues
of the Schur complement $S$ or its components $S_\mathcal{P}, S_\mathcal{B}$.
We see that the eigenvalues drop exponentially if $m$ is increased. If the mass
matrix is used as the proconditioner, the exponential decay is not as rapid.
\begin{figure}
  \begin{center}
    \includegraphics[width=0.75\textwidth]{img/shen_(i).pdf}
  \caption{Shen. Smallest eigenvalues of problem (\ref{eq:S_eig}) with $\Nmat$ as
  a matrix of $L^2(H^0)$ and $H^{-1}$ norms.}
  \label{fig:pois_shen_1}
  \end{center}
\end{figure}
\begin{figure}
  \begin{center}
    \includegraphics[width=0.75\textwidth]{img/shen_(ii).pdf}
  \caption{Shen. Smallest eigenvalues of problem (\ref{eq:Sp_eig}) with
    $\Nmat_\mathcal{P}$ as a matrix of $L^2(H^0)$ and $H^{-1}$ norms.}
  \label{fig:pois_shen_2}
  \end{center}
\end{figure}
\begin{figure}
  \begin{center}
    \includegraphics[width=0.75\textwidth]{img/shen_(iii).pdf}
  \caption{Shen. Smallest eigenvalues of problem (\ref{eq:Sb_eig}) with
    $\Nmat_\mathcal{B}$ as a matrix of $L^2(H^0)$ and $H^{-1}$ norms.}
  \label{fig:pois_shen_3}
  \end{center}
\end{figure}
Figure \ref{fig:pois_shen_4} shows the Babuska constant. We see that the
difference in decay between $\Nmat=\mathbb{I}$ and $\Nmat$ due to $H^{-1}$
norm is much less. This can perhaps be related to the fact that the scaling
of the right-hand side is mainly controlled by the two dimensional stiffness
matrix, which, unlike the in one dimensional case, scales as $\Mmat$, that is,
its condition number grows exponentially.
\begin{figure}
  \begin{center}
    \includegraphics[width=0.75\textwidth]{img/shen_(iv).pdf}
    \caption{Shen.Smallest eigenvalues of problem (\ref{eq:babuska}) with $\Nmat$ as
    a matrix of $L^2(H^0), H^{-1}$ norms and $\Mmat=\Amat$}
  \label{fig:pois_shen_4}
  \end{center}
\end{figure}

\subsection{Finite element method}
\section{Biharmonic problem}
\section{Doodles}
\subsection{Spurious modes}
The physical processes enters the saddle point problem and the eigenvalue
problems through the bilinear form $a$ and the norms $m, n$ which induces norms.
On the discrete level, this translates to matrices $\Amat, \Mmat, \Nmat$. The 
matrix $\Bmat$ remains unchanged. This permits us to discuss the question of
existence of the spurious modes regardless of the physical problem, which is
critical for uniqueness. Note that we are only able to discuss the existence of
such modes. For stability of discretization the important factor is the relative
speed with which we approach the space of spurious modes when the discrete spaces
are inflated. This rate requires norms and therefore cannot be discussed separately.

Recall the condition for uniqueness of the solution to (\ref{system}) read
$\Bmat\Pvec=0\rightarrow\Pvec=0$. The product on the right hand side can
be equivalently written as
\[
  \Bmat\Pvec = \begin{bmatrix}
                \Bp \\
                \Bb
               \end{bmatrix}
               \begin{bmatrix}
                 \Pvec
               \end{bmatrix}
               =\begin{bmatrix}
                \Bp\Pvec \\
                \Bb\Pvec
               \end{bmatrix}.
\]
Therefore $\Bmat\Pvec=0$ only if $\Pvec\in\ker{\Bp}\cap\ker{\Bb}$. Consider
  
% Restriction operator and Trace operator
% Scaling of matrices
% 'Strucrure of spaces' by distacne Vb-Q by distances
%                                   Vp|r by distances


  \bibliographystyle{plain}
  \bibliography{plate_beam}
\end{document}
