\documentclass[a4paper,10pt]{article}
\pdfoptionpdfminorversion=5
\usepackage[utf8]{inputenc}
\usepackage{amsmath}
\usepackage{amssymb}
\usepackage{amsfonts}
\usepackage{multicol}
\usepackage{xcolor,colortbl}
\usepackage{datetime}
\usepackage{xfrac}

\newcommand{\mc}[2]{\multicolumn{#1}{c}{#2}}
\definecolor{Gray}{gray}{0.85}
\newcolumntype{a}{>{\columncolor{Gray}}c}
\newcolumntype{b}{>{\columncolor{white}}c}
\newcolumntype{R}{>{\columncolor{lime}}c}
\newcolumntype{B}{>{\columncolor{pink}}c}
\usepackage{diagbox}

\usepackage[numbers]{natbib}

\usepackage[utf8]{inputenc}
\usepackage{array,multirow,graphicx}
\usepackage{graphics}
\usepackage{color}
\usepackage{ifpdf}
\ifpdf
\DeclareGraphicsRule{*}{mps}{*}{}
\fi
\usepackage[utf8]{inputenc}

\hoffset = 0pt
\voffset = 0pt
\oddsidemargin = 0pt
\headheight = 15pt
\footskip = 30pt
\topmargin = 0pt
\marginparsep = 5pt
\headsep =25pt
\marginparwidth = 54pt
\marginparpush = 7pt
\textheight = 621pt %621/
\textwidth = 500pt

% Default fixed font does not support bold face
\DeclareFixedFont{\ttb}{T1}{txtt}{bx}{n}{12} % for bold
\DeclareFixedFont{\ttm}{T1}{txtt}{m}{n}{12}  % for normal

\newcommand{\I}{\ensuremath{\mathbb{I}}}
\newcommand{\R}{\ensuremath{\mathbb{R}}}

\newcommand{\norm}[1]{\ensuremath{\left\|#1\right\|}}
\newcommand{\dual}[2]{\ensuremath{\langle#1, #2 \rangle}}

\newcommand{\inner}[2]{\ensuremath{\left(#1, #2\right)}}
\newcommand{\Inner}[2]{\ensuremath{\left(\left(#1, #2\right)\right)}}

\newcommand{\deriv}[2]{\ensuremath{\frac{\mathrm{d}#1}{\mathrm{d}#2}}}
\newcommand{\meas}[1]{\ensuremath{\mathrm{d}\, #1}}
\newcommand{\Div}[1] {\ensuremath{\text{div}#1}}
\newcommand{\Grad}[1]{\ensuremath{\text{grad}#1}}
\newcommand{\Curl}[1]{\ensuremath{\text{curl}#1}}
\newcommand{\jump}[1]{\ensuremath{[\![#1]\!]} }


\newcommand{\Vp}{\ensuremath{V_{\mathcal{P}}}}
\newcommand{\Vb}{\ensuremath{V_{\mathcal{B}}}}
\newcommand{\Ep}{\ensuremath{E_{\mathcal{P}}}}
\newcommand{\Eb}{\ensuremath{E_{\mathcal{B}}}}

\newcommand{\W}[1]{\ensuremath{w\!\left[#1\right]\!}}
\newcommand{\E}[1]{\ensuremath{\epsilon \!\left[#1\right]\!}}
\newcommand{\T}[1]{\ensuremath{\sigma \! \left[#1\right]\!}}

\newcommand{\Tr}[1]{\ensuremath{\text{tr}#1}}
\newcommand*{\defeq}{\mathrel{\vcenter{\baselineskip0.5ex \lineskiplimit0pt
                     \hbox{\scriptsize.}\hbox{\scriptsize.}}}%
                     =}

\usepackage{lipsum}
\usepackage{authblk}
\usepackage{fancyhdr}

\DeclareMathOperator{\spn}{span}

% DEBUGGING
\usepackage{lineno}
\linenumbers
%\usepackage{setspace}
%\doublespacing
%
%\pagestyle{fancy}
%
\renewenvironment{abstract}{%
\hfill\begin{minipage}{0.95\textwidth}
\rule{\textwidth}{1pt}}
{\par\noindent\rule{\textwidth}{1pt}\end{minipage}}
%
\makeatletter
\renewcommand\@maketitle{%
\hfill
\begin{minipage}{0.95\textwidth}
\vskip 2em
\let\footnote\thanks 
{\LARGE \@title \par }
\vskip 1.5em
{\large \@author \par}
\end{minipage}
\vskip 1em \par
}
\makeatother
%
\begin{document}
%
%title and author details
\title{\begin{center}
        Note on plate-beam-like problems
       \end{center}}
\author[1]{MK}
%\author[2]{Name2}
%\affil[1]{Address of author} 
%\affil[2]{Address of second author}
%
\maketitle
%
\begin{abstract}
  This report is a summary of observations made on the properties of saddle
  point systems that arise when a physical process in a domain
  $\mathcal{P}\subset\R^2$ is coupled to another physical process in a domain
  $\mathcal{B}\subset\mathcal{P}$ with topological dimension equal to one and
  the coupling constraint does not involve a differential operator. We discuss
  two cases (i) the governing equations of both processes are given by the
  Laplace operator, (ii) the governing equations of both process are given by
  the biharmonic operator.
\end{abstract}

\section{Introduction}
  We consider $\mathcal{P}=\left[-1, 1\right]^2$ and reffer to this domain as
  plate. To define $\mathcal{B}$ we let
  $\vec{\chi}, \vec{\chi}:\left[-1, 1\right]\mapsto\mathcal{P}$ denote some
  invertible mapping from the bi-unit interval onto the plate. Then
  $\mathcal{B}=\left\{\vec{x}\in\R^2, \vec{x}=\vec{\chi}(s), s\in\left[-1,
  1\right]\right\}$. For simplicity we shall have
  $\vec{\chi}(s)=\frac{\vec{A}}{2}(1-s) +\frac{\vec{B}}{2}(1+s)$ for some
  disctinct points $\vec{A}, \vec{B}\in\partial\mathcal{P}$. We then reffer to
  the set $\mathcal{B}$ as beam. Further we let $V_{\mathcal{P}}, V_{\mathcal{B}}$
  be the spaces of functions that map plate and the beam respectivelly to real
  numbers. We postpone discussion of regularity properties of functions in these
  two spaces for later sections.

  The cases considered in this report are special cases of the following
  problem: Find $u\in\Vp, v\in\Vb$ that minimize a functional
  \begin{equation}
    \label{eq:energy}
    \begin{aligned}
      E(v, q) &= \Ep(v, v) + \Eb(q, q) - \int_{\mathcal{P}}f v d\vec{x}, \\
      T(v) &= q\text{ on }\mathcal{B},
    \end{aligned}
  \end{equation}
  where $\Ep, \Ep: \Vp\times\Vp\mapsto\R$ is a bilinear form that describes
  energy of the physical process on the plate and similarly the bilinear form
  $\Eb, \Eb:\Vb\times\Vb\mapsto\R$ describes energy of the beam process. The
  last term in the expression is a potential energy due to forcing $f$. Finaly
  we denote as $T, T:\Vp\mapsto\Vb$ the trace operator.
  
  The problem (\ref{eq:energy}) can be also recast into an unconstrained
  optimization setting, where we extremize Lagrangian
  \begin{equation}
    \label{eq:lagrangian}
    L(v, q, \mu) = \Ep(v, v) + \Eb(q, q) - \int_{\mathcal{P}}f u d\vec{x} -
  \int_{\mathcal{B}}(T(v) - q)\mu d\vec{x}.
  \end{equation}
  Here, the function $\mu$, $\mu\in Q$ is an unknown Lagrange multipler
  and the function space $Q$ has functions that map the beam to real scalars.
  Regularity of functions in $Q$ is discussed later.

  An extreme point of the Lagrangian (\ref{eq:lagrangian}) is given as a
  solution of the problem: Find $u\in\Vp, p\in\Vb, \lambda\in Q$ such that
  for all $v\in\Vp, q\in\Vb, \mu\in Q$ it holds that
 \begin{equation}
    \label{eq:system}
    \begin{aligned}
      \Ep(u, v) - \int_{\mathcal{B}}T(v)\lambda d\vec{x} &= \int_{\mathcal{P}}fvd\vec{x}, \\
      \Eb(p, q) - \int_{\mathcal{B}}q\lambda d\vec{x} &= 0, \\
      \int_{\mathcal{B}}(T(u)-p)\mu d\vec{x} &= 0.
    \end{aligned}
  \end{equation}
  System (\ref{eq:system}) has a unique solution provided that the
  Babuska-Brezzi conditions are satisfied. It is known that even if the
  conditions hold for the tuple of spaces $(\Vp, \Vb, Q)$ it is not quaranteed
  that this property will transfer down to finite-dimensional approximatations
  of the spaces which we denote $V_m, V_n, Q_k$. A famous example of this
  is a Stokes problem whose stable discretization with the finite element method
  requires special pairs of finite elements for velocity and pressure.
  Similarly, polynomials of different order are required for the velocity and
  pressure if the problem is to be solved by spectral methods.

  \section{Plans}

  Problems (i), (ii) are solved by the Galekin method with Fourier basis and
  the basis due to Shen. In the first case, the Fourier and Shen basis have
  same bcs. In the latter one the Fourier bases gives conditions on the function
  value and the value of the second derivative while the Shen basis constrains
  the function value and the first derivative. \textbf{So it would be nice to 
  get Shen like basis so that the boundary conditions are the same}.

  For both (i), (ii) we have numerical experiments with the Fourier basis which
  investigate preconditioner for the Schur complement. The preconditioner is
  related to the norm of $\lambda$ in the LBB condition. As part of the
  preconditioner investigations we can show that the scaling of the preconditioned
  system is independent of the beam position. What seems to change
  with the beam position is the condition number of the preconditioned system.
  A fun question for me is whether stiffer systems mean stiffer plate-beam. 
  \textbf{Similar experiments must be made with Shen basis}. It is possible that
  there the scalling of the differential operator might be more difficult to see.
  Finally at least \textbf{for (i), the above should be possible with FEniCS}.

  Problems (i), (ii) share the constraint and the trace operator is part of the
  LBB condition for both. We can construct a matrix that is related to the trace
  operator with Fourier basis, both Shen basis and perhaps FEniCS. \textbf{We
  could then ask what the properties are of this matrix for different discretization
  parameters, $N, h$ and what are their paralles in case of the trace operator.}

%------------------------------------------------------------------------------

  \section{Galerkin method with the Fourier basis}
  In this section we study convergence properties of the Galerkin method with 
  finite-dimensional approximation spaces constructed using the Fourier basis.
  The method is used to solve one and two dimensional Laplace and biharmonic 
  problems. In all four cases we provide estimates for the convergence rates.
  The rates are then verified by numerical experiments. This should validate
  the method as a viable candidate to be used in our investigations of the
  coupled problems.

  \subsection{One dimensional Poisson problem}
  We consider an eigenvalue problem for the one-dimensional Laplace operator
  $\delta, \delta(u) = -\tfrac{d^2u}{dx^2}$
  \begin{equation}
  \label{eq:eig_pos1}
    \delta(u) = \lambda u \,\text{ in }(-1, 1),
  \end{equation}
  subjected to the homegeneous Dirichlet boundary conditions.
  Defining $\alpha_k = \frac{\pi}{2} + k\frac{\pi}{2}, k\geq0$, we have that the
  eigenfunctions of operator $\delta$ are
  \[
  \phi_k(x) = \begin{cases}
                \sin{\alpha_k x},\,\,k\,\,\text{odd}\\
                \cos{\alpha_k x},\,\,k\,\,\text{even}\\
              \end{cases}
  \]
  and $\lambda_k = \alpha_k^2$ are the eigenvalues. We note that in accordance
  with tha fact that $\delta$ is symmetric and positive definite, the eigevalues
  are real and positive. Morover the eigenfunctions are orthogonal in the $L^2$
  inner product over the biunit interval which we shall denote as
  $\inner{\cdot}{\cdot}$. By the spectral decomposition theorem
  we can then write the operator as
  \[
    \delta = \sum_{k=0}^{\infty} \lambda_k \phi_k \otimes \phi_k.
  \]
  Its inverse is given as
  \begin{equation}
    \label{eq:delta_inv}
    \delta^{-1} = \sum_{k=0}^{\infty} \frac{1}{\lambda_k} \phi_k \otimes \phi_k.
  \end{equation}
  As we know the inverse we can immediately get a solution to the Poisson
  equation
  \begin{equation}
  \label{eq:poisson_strong_1}
  \begin{aligned}
    -&\deriv{^2u}{x^2} = f\,\text{ in }(-1, 1),\\
     &u(-1) = u(1) = 0
  \end{aligned}
  \end{equation}
  as a weighted projection of $f$ onto function space spanned by the
  eigenfunctions
  \begin{equation}
  \label{eq:poisson_1d_sol}
  u = \sum_{k=0}^{\infty} \frac{1}{\lambda_k} \phi_k (f, \phi_k).
  \end{equation}

  To introduce our numerical method we consider a weak formulation of
  (\ref{eq:poisson_strong_1}): Find $u\in V\defeq H^1_0((-1, 1))$ such that
  \[
    \inner{\deriv{u}{x}}{\deriv{v}{x}} = \inner{f}{v},\,\forall v\in V.
  \]
  We now let $V_n=\spn\{\phi_j\}_{j=0}^{n-1}$. We shall refer to the set 
  $\left\{\phi_i\right\}_{i=0}^{\infty}$ as the Fourier basis. The space $V_n$
  is a clearly a finite dimensional subspace of $V$. The weak formulation of
  the Poisson problem on $V_n$ reads: Find $u_n\in V_n$ such that
  \begin{equation}
    \label{eq:poisson_weak_1}
    \inner{\deriv{u_n}{x}}{\deriv{v}{x}} = \inner{f}{v},\,\forall v\in V_n.
  \end{equation}
  Finding the solution to problem (\ref{eq:poisson_weak_1}) is equivalent to
  obtaining the solution of linear system $\mathbb{A}\mathbf{u}=\mathbf{b}$.
  We note that by the properties of basis function of $V_n$, the system matrix
  $\mathbb{A}\in\R^{n\times n}$ is diagonal with $\mathbb{A}_{i, i}=\lambda_i$
  and the $i$-th component of the solution vector is given by
  \[
    \mathbf{u}_i = \frac{\inner{f}{\phi_i}}{\lambda_i}.
  \]
  But then for the solution of (\ref{eq:poisson_weak_1}) we have
  \[
    u = \sum_{k=0}^{n-1} \frac{1}{\lambda_k} \phi_k (f, \phi_k).
  \]
  Comparing with (\ref{eq:poisson_1d_sol}) we see that the error
  \[
    e_n=u-u_n = \sum_{k=n}^{\infty} \frac{1}{\lambda_k} \phi_k (f, \phi_k)
  \]
  such that $(e_n, v)=0$ for all $v\in V_n$. With the exact expression for the
  error we are in a position to establish convergence properties of the
  numerical method. The convergence rate in the $L^2$ norm follows from the
  estimate
  \[
    \norm{e_n}^2=\inner{e_n}{e_n} = \sum_{k=n}^{\infty} \frac{1}{\lambda^2_k}(f, \phi_k)^2 \leq
      \frac{1}{\lambda^2_n}\sum_{k=n}^{\infty} (f, \phi_k)^2 \leq
      \frac{1}{\lambda^2_n}\norm{f}^2.
  \]
  We thus have $\norm{e_n}\leq\frac{1}{\left(\frac{\pi}{2} +
  n\frac{\pi}{2}\right)^2}\norm{f}$ and expect the error to drop as
  $\mathcal{O}(n^{-2})$. To get the convergance rate in the energy norm we use
  the estimate
  \[
    \norm{e_n}^2_1=\inner{\deriv{e_n}{x}}{\deriv{e_n}{x}} =
    \sum_{k=n}^{\infty} \frac{\lambda_k}{\lambda^2_k}(f, \phi_k)^2 \leq
    \frac{1}{\sqrt{\lambda_n}}\sum_{k=n}^{\infty} (f, \phi_k)^2 \leq
    \frac{1}{\sqrt{\lambda_n}}\norm{f}^2.
  \]
  It follows that in the energy norm the error should behave as
  $\mathcal{O}(n^{-1})$.

  To test our convergence estimates we have performed a convergence study where
  the exact solution of (\ref{eq:poisson_strong_1}) was given as as
  $u=(x^2-1)e^x$. The rates in both the $L^2$ norm and the $H^1$ norm are
  summarized in Table \ref{tab:eig_p_1d}. We see that they are slightly higher
  than our estimates but in general agree with analysis.
  \begin{table}
    \centering
    \begin{tabular}{|c|c|c|c||c|c|c|}
    \hline
    & \multicolumn{3}{ c|| }{$L^2$} & \multicolumn{3}{c|}{$H^1$}\\
    \hline
    $n$ & $e$ & $p$ & $E$ & $e$ & $p$ & $E$\\
    \hline
    3 & 7.50E-02 & 1.67 & 1E-21 & 5.48E-01 & 0.97 & 1E-19\\
    4 & 3.98E-02 & 2.20 & 1E-18 & 3.75E-01 & 1.32 & 1E-33\\
    5 & 2.57E-02 & 1.95 & 1E-36 & 2.90E-01 & 1.16 & 1E-33\\
    6 & 1.66E-02 & 2.41 & 1E-36 & 2.23E-01 & 1.44 & 1E-33\\
    7 & 1.21E-02 & 2.06 & 1E-36 & 1.84E-01 & 1.23 & 1E-33\\
    8 & 8.64E-03 & 2.50 & 1E-35 & 1.51E-01 & 1.50 & 1E-31\\
    9 & 6.73E-03 & 2.12 & 1E-32 & 1.30E-01 & 1.27 & 1E-28\\
    10 & 5.14E-03 & 2.55 & 1E-29 & 1.11E-01 & 1.53 & 1E-24\\
    11 & 4.19E-03 & 2.15 & 1E-27 & 9.77E-02 & 1.29 & 1E-22\\
    12 & 3.35E-03 & 2.59 & 1E-24 & 8.54E-02 & 1.55 & 1E-19\\
    13 & 2.81E-03 & 2.17 & 1E-21 & 7.70E-02 & 1.30 & 1E-17\\
    14 & 2.32E-03 & 2.61 & 1E-17 & 6.85E-02 & 1.57 & 1E-34\\
    15 & 1.99E-03 & 2.19 & 1E-38 & 6.26E-02 & 1.31 & 1E-35\\
    \hline
    \hline
    & \multicolumn{3}{ c|| }{2.19} & \multicolumn{3}{c|}{1.31}\\
    \hline
    \end{tabular}
    \label{tab:eig_p_1d}
    \caption{Convergence rate of the Fourier-Galerkin method for one dimensional
    Poisson problem. For each norm we list as $e$ the magnitude of error,
    $p$ the convergence rate while $E$ is the estimate of the quadrature error
    in the computation of $e$ provided by SymPy's {\tt{quad}} function. The last
    row in the table has the estimate for the rate obtained by least-squares fit.}
  \end{table}

  \subsection{One dimensional biharmonic problem}
  As in the previous section, in order to solve the one dimensional biharmonic
  problem
  \begin{equation}
  \label{eq:bih_strong_1}
  \begin{aligned}
    &\deriv{^4u}{x^4} = f\,\text{ in }(-1, 1),\\
    &u(-1) = u(1) = 0,\\
    &\deriv{^2u}{x^2}(-1) = \deriv{^2u}{x^2}(1) = 0
  \end{aligned}
  \end{equation}
  we shall first consider the corresponding eigenvalue problem
  \begin{equation}
  \begin{aligned}
    &\deriv{^4u}{x^4} = \lambda u,\text{ in }(-1, 1),\\
    &u(-1) = u(1) = 0,\\
    &\deriv{^2u}{x^2}(-1) = \deriv{^2u}{x^2}(1) = 0.
  \end{aligned}
  \end{equation}
  With the given boundary conditions, the operator acting on the left hand
  side is $\delta^2$. It thus has the same eigenfunctions as $\delta$ while
  the eigenvalues are squares of the eiqenvalues of $\delta$, that is
  $\lambda_k=\alpha_k^4$.
  
  Our numerical method for solving (\ref{eq:bih_strong_1}) is based on the 
  weak formulation of the problem which reads: Find $u\in V\defeq\left\{v\in
  H^2((-1, 1)), u(x)=\deriv{^2u}{x^2}(x),\text{for}\,x=\pm 1\right\}$ such that
  \[
    \inner{\deriv{^2u}{x^2}}{\deriv{^2v}{x^2}} = \inner{f}{v},\,\forall v\in V.
  \]
  The space $V_n$ introduced in the previous section is the subspace of $V$
  and it is therefore valid to consider a problem: Find $u_n\in V_n$ such that
  \begin{equation}
    \label{eq:bih_weak_1}
    \inner{\deriv{^2u}{x^2}}{\deriv{^2v}{x^2}} = \inner{f}{v},\,\forall v\in V_n.
  \end{equation}
  We find that the corresponding linear system $\mathbb{A}\mathbf{u}=\mathbf{b}$
  is solved by vector $\mathbf{u}\in\R^n$ with components
  $\mathbf{u}_i=\frac{\mathbf{b}_i}{\alpha_i^4}$ and consequently the error of the
  method can be axpressed as
  \[
    e_n = \sum_{k=n}^{\infty} \frac{1}{\alpha_k^4} \phi_k (f, \phi_k).
  \]

  Convergence of the method in the $L^2$ norm follows from the estimate
  \[
    \norm{e_n}^2=\inner{e_n}{e_n} = \sum_{k=n}^{\infty} \frac{1}{\alpha^8_k}(f, \phi_k)^2 \leq
      \frac{1}{\alpha^8_n}\sum_{k=n}^{\infty} (f, \phi_k)^2 \leq
      \frac{1}{\alpha^8_n}\norm{f}^2
  \]
  which yields $\norm{e}\leq\frac{1}{\left(\frac{\pi}{2} +
  n\frac{\pi}{2}\right)^4}\norm{f}$. We have $\norm{e}$ decreasing as
  $\mathcal{O}(n^{-4})$. Since each differentiation of the eigenfunction 
  $\phi_k$ produces a factor $\alpha_k$ we get the estimates of the error in 
  $H^1$ and $H^2$ norms as
  \begin{align*}
    \norm{e_n}_1 &\leq \frac{1}{\alpha^3_n}\norm{f}, \\
    \norm{e_n}_2 &\leq \frac{1}{\alpha^2_n}\norm{f}. \\
  \end{align*}
  The errors in the respected norms should therefore behave as
  $\mathcal{O}(n^{-3})$ and $\mathcal{O}(n^{-2})$.
    
  Table \ref{tab:eig_b_1d} shows results from the convergence study we run to
  verify convergence properties of the method established in the previous
  pragraphs. In this study, the exact solution of (\ref{eq:bih_strong_1}) was
  computed to match the right hand side $f=x e^x$. The results show very good
  agreement with the theoretical estimates.

  \begin{table}
    \centering
    \begin{tabular}{|c|c|c|c||c|c|c||c|c|c|}
    \hline
    & \multicolumn{3}{c||}{$L^2$} &
      \multicolumn{3}{c||}{$H^1$} &
      \multicolumn{3}{c|}{$H^2$}\\
    \hline
$n$ & $e$ & $p$ & $E$ & $e$ & $p$ & $E$ & $e$ & $p$ & $E$\\
\hline
3 & 3.08E-04 & 2.68 & 1E-27 & 2.00E-03 & 2.01 & 1E-25 & 1.33E-02 & 1.37 & 1E-25\\
4 & 8.63E-05 & 4.42 & 1E-21 & 7.28E-04 & 3.50 & 1E-19 & 6.41E-03 & 2.54 & 1E-21\\
5 & 4.42E-05 & 3.00 & 1E-20 & 4.38E-04 & 2.28 & 1E-19 & 4.49E-03 & 1.60 & 1E-18\\
6 & 1.81E-05 & 4.90 & 1E-34 & 2.16E-04 & 3.86 & 1E-41 & 2.70E-03 & 2.79 & 1E-38\\
7 & 1.12E-05 & 3.10 & 1E-35 & 1.50E-04 & 2.38 & 1E-41 & 2.09E-03 & 1.68 & 1E-39\\
8 & 5.65E-06 & 5.14 & 1E-37 & 8.76E-05 & 4.03 & 1E-41 & 1.42E-03 & 2.90 & 1E-38\\
9 & 3.90E-06 & 3.15 & 1E-36 & 6.58E-05 & 2.42 & 1E-39 & 1.16E-03 & 1.72 & 1E-36\\
10 & 2.24E-06 & 5.27 & 1E-17 & 4.26E-05 & 4.13 & 1E-33 & 8.47E-04 & 2.96 & 1E-32\\
11 & 1.65E-06 & 3.17 & 1E-33 & 3.37E-05 & 2.45 & 1E-32 & 7.17E-04 & 1.74 & 1E-30\\
12 & 1.04E-06 & 5.36 & 1E-28 & 2.34E-05 & 4.19 & 1E-26 & 5.53E-04 & 3.00 & 1E-25\\
13 & 8.03E-07 & 3.19 & 1E-26 & 1.92E-05 & 2.47 & 1E-24 & 4.80E-04 & 1.76 & 1E-22\\
14 & 5.38E-07 & 5.42 & 1E-22 & 1.41E-05 & 4.23 & 1E-20 & 3.84E-04 & 3.03 & 1E-18\\
15 & 4.31E-07 & 3.21 & 1E-21 & 1.18E-05 & 2.48 & 1E-19 & 3.40E-04 & 1.77 & 1E-17\\
\hline
    \hline
    & \multicolumn{3}{c||}{3.93} &
      \multicolumn{3}{c||}{3.05} &
      \multicolumn{3}{c|}{2.18}\\
    \hline
    \end{tabular}
    \label{tab:eig_b_1d}
    \caption{Convergence rate of the Fourier-Galerkin method for one dimensional
    biharmonic problem. For each norm we list as $e$ the magnitude of error,
    $p$ the convergence rate while $E$ is the estimate of the quadrature error
    in the computation of $e$ provided by SymPy's {\tt{quad}} function. The last
    row in the table has the estimate for the rate obtained by least-squares fit.}
  \end{table}

  We have demonstrated by both theoretical estimates and the convergence studies
  that the presented Fourier-Galerkin method can be used to solve accurately 
  the one dimensional Poisson and biharmonic problems. We therefore have at our
  disposal a numerical method to resolve the beam physics of cases (i) and (ii).
  In the next sections we show how the Fourier basis can be used to solve
  Poisson and biharmonic problems in two dimensional domain, that is to resolve
  the physical processes on the beam considered in this note.

  \subsection{Two dimensional Poisson problem}
  We are interested in finding a solution to the Poisson problem on a bi-unit
  square with homogeneous Dirichlet boundary conditions on the boundary of the
  square
  \begin{equation}
  \label{eq:poisson_strong_2}
  \begin{aligned}
    -\Delta u &= f\,\text{ in }\mathcal{P},\\
            u &= 0\,\text{ on }\mathcal{P}.
  \end{aligned}
  \end{equation}
  A variational formulation of (\ref{eq:poisson_strong_2}) reads: Find $u\in
  V\defeq H^1_0\left(\mathcal{P}\right)$ such that for all $v\in V$ it holds
  that
  \begin{equation}
    \label{eq:poisson_weak_2}
    \Inner{\Grad{u}}{\Grad{v}} = \Inner{f}{v},
  \end{equation}
  where $\Inner{\cdot}{\cdot}$ denotes the $L^2$ inner producut over $\mathcal{P}$.
  We define $V_n$, a finite dimensional space in which the approximation of $u$
  is sought as $V_n=\spn\left\{\phi_i(x)\phi_j(y)\right\}_{i, j = 0, 0}^{n-1, n-1}$.
  Clearly, $V_n\subset V$ and $\text{dim}V_n = n^2$. Moreover, each basis function of
  $\phi_{i, j}, \phi_{i, j}(x, y)\defeq\phi_i(x)\phi_j(y)$, is an eigenfunction
  of the Laplace operator acting on functions in $\mathcal{P}$ with zero boundary
  values. We have that $\lambda_{i, j}=\lambda_i + \lambda_j$, where $\lambda_i$
  is the $i$-th eigenvalue of $\delta$. Finally we note that the basis functions
  of $V_n$ are orthonormal in the $\Inner{\cdot}{\cdot}$ inner product.
  Considering the problem (\ref{eq:poisson_strong_2}) we get that the expansion
  coefficients $\mathbb{U}\in\R^{n\times n}$ of the approximation of $u$ in $V_n$
  \[
    u_n=\sum\limits_{i, j = 0, 0}^{n-1, n-1}\mathbb{U}_{i,
    j}\phi_{i, j}
  \]
  are given as a solution of the linear system
  \begin{equation}
    \label{eq:p_sys}
    \Lambda\mathbb{U} + \Lambda\mathbb{U} = \mathbb{B},
  \end{equation}
  where the matrix $\mathbb{B}\in\R^{n\times n}$ has $\mathbb{B}_{i,
  j}=\Inner{v}{\phi_{i, j}}$. Further, matrix $\Lambda\in\R^{n\times n}$ is
  a diagonal matrix with diagonal elements $\Lambda_{i, i}$ equal to the
  eigenvalues $\lambda_i$. We note that if space $V_n$ were constructed from 
  functions other then eigenfunctions of the Laplacian, e.g. polynomials, the
  Galerkin method would lead to the system
  \[
    \tilde{\mathbb{A}}\mathbb{U}\tilde{\mathbb{M}} +
    \tilde{\mathbb{M}}\mathbb{U}\tilde{\mathbb{A}} =
    \tilde{\mathbb{B}},
  \]
  with $\tilde{\mathbb{A}}$ and $\tilde{\mathbb{M}}$ being matrices of one
  dimensional Laplace and identity operators with respect to the chosen basis.
  The solution would then be obtained using transformation matrices $\mathbb{Q}$
  with columns from vectors solving a generalized matrix eigenvalue problem
  $\tilde{\mathbb{A}}q=\alpha\tilde{\mathbb{M}}q$.
  In this setting, the choice of the Fourier basis presents a major advantage
  since it yields mass matrix which is identity and owing to
  $\mathbb{Q}=\mathbb{I}$ the matrix $\Lambda$ is precisely
  $\mathbb{Q}^{\text{T}}\tilde{\mathbb{A}}\mathbb{Q}$. In fact the solution
  of (\ref{eq:p_sys}) is given as
  \[
    \mathbb{U}_{i, j} = \frac{\mathbb{B}_{i, j}}{\lambda_i + \lambda_j}.
  \]
  Consequently, the approximation $u_n$ takes the form
  \[
    u_n = \sum\limits_{i, j = 0, 0}^{n-1, n-1}\frac{\Inner{f}{\phi_{i, j}}}{\lambda_i + \lambda_j}
    \phi_{i, j}
  \]
  which should be contrasted with the exact solution
  \[
    u = \sum\limits_{i, j = 0, 0}^{\infty, \infty}\frac{\Inner{f}{\phi_{i, j}}}{\lambda_i + \lambda_j}
    \phi_{i, j}.
  \]
  As in the one dimensional case we see that the numerical solution due to
  Fourier-Galerkin method is $u_n = -\Delta|_{n}^{-1}f$ where
  $-\Delta|_{n}^{-1}=P_n \circ \left(-\Delta^{-1}\right)$ is a projection by 
  the operator $P_n, P_n:V\mapsto V_n$ of the inverse of the Laplacian on the
  subspace $V_n$.

  The convergence rate of the method in the $L^2$ and the energy norm follow
  from the estimates which are obtained using same ideas as in the one
  dimensional case
  \[
    \norm{e}^2 = \sum\limits_{i, j = n, n}^{\infty,
    \infty}\frac{\Inner{f}{\phi_{i, j}}^2}{\left(\lambda_i + \lambda_j\right)^2}
      \leq
      \frac{1}{\left(2\lambda_n\right)^2}\norm{f}^2
  \]
  and
\[
    \norm{e}_1^2 = \sum\limits_{i, j = n, n}^{\infty,
    \infty}\frac{\Inner{f}{\phi_{i, j}}^2}{\left(\lambda_i + \lambda_j\right)}
      \leq
      \frac{1}{\left(2\lambda_n\right)}\norm{f}^2
  \]
  We thus have $\norm{e}\leq\frac{1}{\left(2\pi + 2n\pi\right)^2}\norm{f}$ and
  $\norm{e}_1\leq\frac{1}{2\pi + 2n\pi}\norm{f}$.

  The estimates were tested by a convergence study with the exact solution
  of \ref{eq:poisson_strong_2} given as $u=(x^2 - 1)e^x(y^2 - 1)$. We note that
  as for the one dimensional Poisson problem theoretical estimates seem to
  overestimate the numerical rates.

  \begin{table}
    \centering
    \begin{tabular}{|c|c|c|c||c|c|c|}
    \hline
    & \multicolumn{3}{ c|| }{$L^2$} & \multicolumn{3}{c|}{$H^1$}\\
    \hline
      $n$ & $e$ & $p$ & $E$ & $e$ & $p$ & $E$\\
      \hline
      3 & 7.82E-02 & 1.75 & 1E-21 & 5.87E-01 & 1.07 & 1E-19\\
      4 & 4.24E-02 & 2.12 & 1E-18 & 4.04E-01 & 1.30 & 1E-33\\
      5 & 2.69E-02 & 2.05 & 1E-36 & 3.07E-01 & 1.24 & 1E-34\\
      6 & 1.76E-02 & 2.32 & 1E-36 & 2.37E-01 & 1.40 & 1E-33\\
      7 & 1.26E-02 & 2.16 & 1E-36 & 1.94E-01 & 1.30 & 1E-33\\
      8 & 9.16E-03 & 2.41 & 1E-35 & 1.60E-01 & 1.45 & 1E-31\\
      9 & 7.05E-03 & 2.21 & 1E-32 & 1.37E-01 & 1.33 & 1E-28\\
      10 & 5.45E-03 & 2.46 & 1E-29 & 1.17E-01 & 1.48 & 1E-24\\
      11 & 4.39E-03 & 2.25 & 1E-27 & 1.03E-01 & 1.35 & 1E-22\\
      12 & 3.54E-03 & 2.49 & 1E-24 & 9.04E-02 & 1.50 & 1E-19\\
      13 & 2.95E-03 & 2.27 & 1E-21 & 8.10E-02 & 1.36 & 1E-17\\
      14 & 2.45E-03 & 2.51 & 1E-17 & 7.24E-02 & 1.51 & 1E-35\\
      15 & 2.09E-03 & 2.29 & 1E-39 & 6.59E-02 & 1.37 & 1E-35\\
      \hline
    \hline
    \hline
    & \multicolumn{3}{ c|| }{2.19} & \multicolumn{3}{c|}{1.33}\\
    \hline
    \end{tabular}
    \label{tab:eig_p_2d}
    \caption{Convergence rate of the Fourier-Galerkin method for two dimensional
    Poisson problem. For each norm we list as $e$ the magnitude of error,
    $p$ the convergence rate while $E$ is the estimate of the quadrature error
    in the computation of $e$ provided by SymPy's {\tt{quad}} function. The last
    row in the table has the estimate for the rate obtained by least-squares fit.}
  \end{table}

  \subsection{Two dimensional biharmonic problem}
  % Finally beharmonic operator section should follow same straucture
  % results as plot
  % We have method for physics of plate problems.
  % The properties are not due to sines cosis but beacuse we have the
  % eigenvectors!
  \subsection{Galerkin method with the basis of Legendre polynomials for the
  Poisson problem}
  % Shoortly introduce Shen's method and verify our implementation for 1d adn
  % 2d problem. 
  % Condition numbers of A, M in both cases
  % Can this be tied to rate?
  % Why don't we see scaling of the differential operator in the Shen case
  % results as plot

  % If you could come up with u, u`` basis with Leg, it could go here and
  % we'd again show plots with rates.
 
\end{document}
