\documentclass[a4paper,10pt]{article}
\pdfoptionpdfminorversion=5
\usepackage[utf8]{inputenc}
\usepackage{amsmath}
\usepackage{amssymb}
\usepackage{amsfonts}
\usepackage{multicol}
\usepackage{xcolor,colortbl}
\usepackage{datetime}
\usepackage{xfrac}

\newcommand{\mc}[2]{\multicolumn{#1}{c}{#2}}
\definecolor{Gray}{gray}{0.85}
\newcolumntype{a}{>{\columncolor{Gray}}c}
\newcolumntype{b}{>{\columncolor{white}}c}
\newcolumntype{R}{>{\columncolor{lime}}c}
\newcolumntype{B}{>{\columncolor{pink}}c}
\usepackage{diagbox}

\usepackage[numbers]{natbib}

\usepackage[utf8]{inputenc}
\usepackage{array,multirow,graphicx}
\usepackage{graphics}
\usepackage{color}
\usepackage{ifpdf}
\ifpdf
\DeclareGraphicsRule{*}{mps}{*}{}
\fi
\usepackage[utf8]{inputenc}

\hoffset = 0pt
\voffset = 0pt
\oddsidemargin = 0pt
\headheight = 15pt
\footskip = 30pt
\topmargin = 0pt
\marginparsep = 5pt
\headsep =25pt
\marginparwidth = 54pt
\marginparpush = 7pt
\textheight = 621pt %621/
\textwidth = 500pt

% Default fixed font does not support bold face
\DeclareFixedFont{\ttb}{T1}{txtt}{bx}{n}{12} % for bold
\DeclareFixedFont{\ttm}{T1}{txtt}{m}{n}{12}  % for normal

\newcommand{\I}{\ensuremath{\mathbb{I}}}
\newcommand{\R}{\ensuremath{\mathbb{R}}}

\newcommand{\norm}[1]{\ensuremath{\left\|#1\right\|}}
\newcommand{\seminorm}[1]{\ensuremath{\left|#1\right|}}
\newcommand{\dual}[2]{\ensuremath{\langle#1, #2 \rangle}}

\newcommand{\inner}[2]{\ensuremath{\left(#1, #2\right)}}
\newcommand{\Inner}[2]{\ensuremath{\left(\left(#1, #2\right)\right)}}

\newcommand{\deriv}[2]{\ensuremath{\frac{\mathrm{d}#1}{\mathrm{d}#2}}}
\newcommand{\meas}[1]{\ensuremath{\,\mathrm{d}#1}}
\newcommand{\Div}[1] {\ensuremath{\text{div}#1}}
\newcommand{\Grad}[1]{\ensuremath{\text{grad}#1}}
\newcommand{\Curl}[1]{\ensuremath{\text{curl}#1}}
\newcommand{\jump}[1]{\ensuremath{[\![#1]\!]} }


\newcommand{\Vp}{\ensuremath{V_{\mathcal{P}}}}
\newcommand{\Vb}{\ensuremath{V_{\mathcal{B}}}}
\newcommand{\Ep}{\ensuremath{E_{\mathcal{P}}}}
\newcommand{\Eb}{\ensuremath{E_{\mathcal{B}}}}

\newcommand{\W}[1]{\ensuremath{w\!\left[#1\right]\!}}
\newcommand{\E}[1]{\ensuremath{\epsilon \!\left[#1\right]\!}}
\newcommand{\T}[1]{\ensuremath{\sigma \! \left[#1\right]\!}}

\newcommand{\Tr}[1]{\ensuremath{\text{tr}#1}}
\newcommand*{\defeq}{\mathrel{\vcenter{\baselineskip0.5ex \lineskiplimit0pt
                     \hbox{\scriptsize.}\hbox{\scriptsize.}}}%
                     =}

\usepackage{lipsum}
\usepackage{authblk}
\usepackage{fancyhdr}

\DeclareMathOperator{\spn}{span}

% DEBUGGING
\usepackage{lineno}
\linenumbers
%\usepackage{setspace}
%\doublespacing
%
%\pagestyle{fancy}
%
\renewenvironment{abstract}{%
\hfill\begin{minipage}{0.95\textwidth}
\rule{\textwidth}{1pt}}
{\par\noindent\rule{\textwidth}{1pt}\end{minipage}}
%
\makeatletter
\renewcommand\@maketitle{%
\hfill
\begin{minipage}{0.95\textwidth}
\vskip 2em
\let\footnote\thanks 
{\LARGE \@title \par }
\vskip 1.5em
{\large \@author \par}
\end{minipage}
\vskip 1em \par
}
\makeatother
%
\begin{document}
%
%title and author details
\title{\begin{center}
        Note on plate-beam-like problems
       \end{center}}
\author[1]{MK}
%\author[2]{Name2}
%\affil[1]{Address of author} 
%\affil[2]{Address of second author}
%
\maketitle
%
\begin{abstract}
  This report is a summary of observations made on the properties of saddle
  point systems that arise when a physical process in a domain
  $\mathcal{P}\subset\R^2$ is coupled to another physical process in a domain
  $\mathcal{B}\subset\mathcal{P}$ with topological dimension equal to one and
  the coupling constraint does not involve a differential operator. We discuss
  two cases (i) the governing equations of both processes are given by the
  Laplace operator, (ii) the governing equations of both process are given by
  the biharmonic operator.
\end{abstract}

\section{Introduction}
  /We consider $\mathcal{P}=\left[-1, 1\right]^2$ and refer to this domain as
  plate. To define $\mathcal{B}$ we let
  $\vec{\chi}, \vec{\chi}:\left[-1, 1\right]\mapsto\mathcal{P}$, denote some
  invertible mapping from the bi-unit interval onto the plate. Then
  $\mathcal{B}=\left\{\vec{x}\in\R^2, \vec{x}=\vec{\chi}(s), s\in\left[-1,
  1\right]\right\}$. For simplicity we shall have
  $\vec{\chi}(s)=\frac{\vec{A}}{2}(1-s) +\frac{\vec{B}}{2}(1+s)$ for some
  distinct points $\vec{A}, \vec{B}\in\partial\mathcal{P}$. We then refer to
  the set $\mathcal{B}$ as beam. Further we let $V_{\mathcal{P}}, V_{\mathcal{B}}$
  be the spaces of functions that map plate and the beam respectively to real
  numbers. We postpone discussion of regularity properties of functions in these
  two spaces for later sections.

  The cases considered in this report are special cases of the following
  problem: Find $u\in\Vp, v\in\Vb$ that minimize a functional
  \begin{equation}
    \label{eq:energy}
    \begin{aligned}
      E(v, q) &= \Ep(v, v) + \Eb(q, q) - \int_{\mathcal{P}}f v \meas{x}\meas{y}, \\
      \text{subject to the constraint}&\\
      T(v) &= q\text{ on }\mathcal{B},
    \end{aligned}
  \end{equation}
  where $\Ep, \Ep: \Vp\times\Vp\mapsto\R$, is a bilinear form that describes
  energy of the physical process on the plate and similarly the bilinear form
  $\Eb, \Eb:\Vb\times\Vb\mapsto\R$, describes energy of the beam process. The
  last term in the expression is a potential energy due to forcing $f$. Finally
  we denote as $T, T:\Vp\mapsto\Vb$ the trace operator.
  
  The problem (\ref{eq:energy}) can be also recast into an unconstrained
  optimization setting, where we extremize Lagrangian
  \begin{equation}
    \label{eq:lagrangian}
    L(v, q, \mu) = \Ep(v, v) + \Eb(q, q) - \int_{\mathcal{P}}f u \meas{x}\meas{y} -
    \int_{\mathcal{B}}(T(v) - q)\mu \meas{x}.
  \end{equation}
  Here, the function $\mu$, $\mu\in Q$, is an unknown Lagrange multiplier
  and the function space $Q$ has functions that map the beam to real scalars.
  Regularity of functions in $Q$ is discussed later.

  An extreme point of the Lagrangian (\ref{eq:lagrangian}) is given as a
  solution of the problem: Find $u\in\Vp, p\in\Vb, \lambda\in Q$ such that
  for all $v\in\Vp, q\in\Vb, \mu\in Q$ it holds that
 \begin{equation}
    \label{eq:system}
    \begin{aligned}
      \Ep(u, v) - \int_{\mathcal{B}}T(v)\lambda\meas{x} &=
      \int_{\mathcal{P}}fv\meas{x}\meas{y}, \\
      \Eb(p, q) - \int_{\mathcal{B}}q\lambda\meas{x} &= 0, \\
      \int_{\mathcal{B}}(T(u)-p)\mu\meas{x}\meas{y} &= 0.
    \end{aligned}
  \end{equation}
  System (\ref{eq:system}) has a unique solution provided that the
  Babuska-Brezzi conditions are satisfied. It is known that even if the
  conditions hold for the tuple of spaces $(\Vp, \Vb, Q)$ it is not quaranteed
  that this property will transfer down to finite-dimensional approximatations
  of the spaces which we denote $V_m, V_n, Q_k$. A famous example of this
  is a Stokes problem whose stable discretization with the finite element method
  requires special pairs of finite elements for velocity and pressure.
  Similarly, polynomials of different order are required for the velocity and
  pressure if the problem is to be solved by spectral methods.

  \section{Plans}

  Problems (i), (ii) are solved by the Galekin method with Fourier basis and
  the basis due to Shen. In the first case, the Fourier and Shen basis have
  same bcs. In the latter one the Fourier bases gives conditions on the function
  value and the value of the second derivative while the Shen basis constrains
  the function value and the first derivative. \textbf{So it would be nice to 
  get Shen like basis so that the boundary conditions are the same}.

  For both (i), (ii) we have numerical experiments with the Fourier basis which
  investigate preconditioner for the Schur complement. The preconditioner is
  related to the norm of $\lambda$ in the LBB condition. As part of the
  preconditioner investigations we can show that the scaling of the preconditioned
  system is independent of the beam position. What seems to change
  with the beam position is the condition number of the preconditioned system.
  A fun question for me is whether stiffer systems mean stiffer plate-beam. 
  \textbf{Similar experiments must be made with Shen basis}. It is possible that
  there the scalling of the differential operator might be more difficult to see.
  Finally at least \textbf{for (i), the above should be possible with FEniCS}.

  Problems (i), (ii) share the constraint and the trace operator is part of the
  LBB condition for both. We can construct a matrix that is related to the trace
  operator with Fourier basis, both Shen basis and perhaps FEniCS. \textbf{We
  could then ask what the properties are of this matrix for different discretization
  parameters, $N, h$ and what are their paralles in case of the trace operator.}

%------------------------------------------------------------------------------

  \section{Galerkin method with the Fourier basis}
  In this section we study convergence properties of the Galerkin method with 
  finite-dimensional approximation spaces constructed using eigenfunctions of
  undelying differential operators. The method is used to solve one and two
  dimensional Poisson and biharmonic problems. In all of these problems the 
  eigenfunctions are sines and cosines, hence the name of the method. We show 
  that the basis of eigenfunctions leads to a very efficient numerical method
  and in all four cases we provide estimates for the convergence rates.
  The rates are then verified by numerical experiments. This should validate
  the method as a viable candidate to be used in our investigations of the
  coupled problems.

  \subsection{One dimensional Poisson problem}
  We consider an eigenvalue problem for the one-dimensional Laplace operator
  $\delta, \delta(u) = -\tfrac{d^2u}{dx^2}$
  \begin{equation}
  \label{eq:eig_pos1}
    \delta(u) = \lambda u \,\text{ in }(-1, 1),
  \end{equation}
  subjected to the homogeneous Dirichlet boundary conditions.
  Defining $\alpha_k = \frac{\pi}{2} + k\frac{\pi}{2}, k\geq0$, we have that the
  eigenfunctions of operator $\delta$ are
  \[
  \phi_k(x) = \begin{cases}
                \sin{\alpha_k x},\,\,k\,\,\text{odd}\\
                \cos{\alpha_k x},\,\,k\,\,\text{even}\\
              \end{cases}
  \]
  and $\lambda_k = \alpha_k^2$ are the eigenvalues. We note that in accordance
  with tha fact that $\delta$ is symmetric and positive definite, the eigenvalues
  are real and positive. Moreover the eigenfunctions are orthonormal in the $L^2$
  inner product over the bi-unit interval which we shall denote as
  $\inner{\cdot}{\cdot}$. By the spectral decomposition theorem
  we can then write the operator as
  \[
    \delta = \sum_{k=0}^{\infty} \lambda_k \phi_k \otimes \phi_k.
  \]
  Next we consider operator $\gamma$
  \[
    \gamma = \sum_{l=0}^{\infty} \frac{1}{\lambda_l} \phi_l \otimes \phi_l
  \]
  and observe that for $v\in V$
  \[
    \gamma\left(\delta v \right) = 
    \sum_{l=0}^{\infty} \frac{1}{\lambda_l}
    \inner{\phi_l}{\sum_{k=0}^{\infty} \lambda_k\inner{\phi_k}{v}\phi_k}\phi_l=
    \sum_{l=0}^{\infty}\sum_{k=0}^{\infty}\inner{v}{\phi_k}\inner{\phi_k}{\phi_l}\frac{\lambda_k}{\lambda_l}\phi_l =
    \sum_{l=0}^{\infty}\inner{v}{\phi_l}\phi_l = 
    v.
  \] 
  Similarly $\delta\left(\gamma v\right)=v$. We thus have that have $\gamma$ is
  the inverse of $\delta$. As we know the inverse we can immediately get a 
  solution to the Poisson equation
  \begin{equation}
  \label{eq:poisson_strong_1}
  \begin{aligned}
    -&\deriv{^2u}{x^2} = f\,\text{ in }(-1, 1),\\
     &u(-1) = u(1) = 0
  \end{aligned}
  \end{equation}
  as a weighted projection of $f$ onto function space spanned by the
  eigenfunctions
  \begin{equation}
  \label{eq:poisson_1d_sol}
  u = \sum_{k=0}^{\infty} \frac{\inner{f}{\phi_k}}{\lambda_k} \phi_k.
  \end{equation}

  To introduce our numerical method we consider a weak formulation of
  (\ref{eq:poisson_strong_1}): Find $u\in V\defeq H^1_0((-1, 1))$ such that
  \[
    \inner{\deriv{u}{x}}{\deriv{v}{x}} = \inner{f}{v},\,\forall v\in V.
  \]
  We now let $V_n=\spn\{\phi_j\}_{j=0}^{n-1}$. The space $V_n$
  is a clearly a finite dimensional subspace of $V$. The weak formulation of
  the Poisson problem on $V_n$ reads: Find $u_n\in V_n$ such that
  \begin{equation}
    \label{eq:poisson_weak_1}
    \inner{\deriv{u_n}{x}}{\deriv{v}{x}} = \inner{f}{v},\,\forall v\in V_n.
  \end{equation}
  Finding the solution to problem (\ref{eq:poisson_weak_1}) is equivalent to
  obtaining the solution of linear system $\mathbb{A}\mathbf{u}=\mathbf{b}$.
  We note that by the properties of basis function of $V_n$, the system matrix
  $\mathbb{A}\in\R^{n\times n}$ is diagonal with $\mathbb{A}_{i, i}=\lambda_i$
  and the $i$-th component of the solution vector is given by
  \[
    \mathbf{u}_i = \frac{\inner{f}{\phi_i}}{\lambda_i}.
  \]
  But then for the solution of (\ref{eq:poisson_weak_1}) we have
  \[
    u_n = \sum_{k=0}^{n-1} \frac{\inner{f}{\phi_k}}{\lambda_k} \phi_k.
  \]
  Comparing with (\ref{eq:poisson_1d_sol}) we see that the error
  \[
    e_n=u-u_n = \sum_{k=n}^{\infty} \frac{\inner{f}{\phi_k}}{\lambda_k} \phi_k
  \]
  such that $(e_n, v)=0$ for all $v\in V_n$. With the exact expression for the
  error we are in a position to establish convergence properties of the
  numerical method. The convergence rate in the $L^2$ norm follows from the
  estimate
  \[
    \norm{e_n}^2=\inner{e_n}{e_n} = \sum_{k=n}^{\infty} \frac{\inner{f}{\phi_k}^2}{\lambda^2_k} \leq
    \frac{1}{\lambda^2_n}\sum_{k=n}^{\infty} \inner{f}{\phi_k}^2 \leq
    \frac{\norm{f}^2}{\lambda^2_n},
  \]
  where it was assumed that $f\in L^2\left(\mathcal{P}\right)$ and used the 
  Parseval inequality. We thus have $\norm{e_n}\leq\frac{4\norm{f}}{\left(\pi +
  n\pi\right)^2}$ and expect the error to drop as
  $n^{-p}, p=2$. We have by the Poincare inequality on $V$ that the $H^1$ seminorm
  defines the so called energy norm $\norm{\cdot}_1$. The convergence rate in
  the energy norm is based on the estimate
  \[
    \norm{e_n}^2_1=\inner{\deriv{e_n}{x}}{\deriv{e_n}{x}} =
    \sum_{k=n}^{\infty} \frac{\lambda_k}{\lambda^2_k}(f, \phi_k)^2 \leq
    \frac{1}{\sqrt{\lambda_n}}\sum_{k=n}^{\infty} (f, \phi_k)^2 \leq
    \frac{\norm{f}^2}{\sqrt{\lambda_n}}.
  \]
  It follows that in the energy norm the error should behave as $n^{-1}$.

  To test our convergence estimates we have performed a convergence study where
  the exact solution of (\ref{eq:poisson_strong_1}) was given as as
  $u=(x^2-1)e^x$. The rates in both the $L^2$ norm and the $H^1$ norm are
  summarized in Table \ref{tab:eig_p_1d}. We see that they are slightly higher
  than our estimates but in general agree with analysis.
  \begin{table}
    \centering
    \begin{tabular}{|c|c|c|c||c|c|c|}
    \hline
      $n$ & $\norm{e}$ & $p$ & $E$ & $\norm{e}_1$ & $p$ & $E$\\
    \hline
    3 & 7.50E-02 & 1.67 & 1E-21 & 5.48E-01 & 0.97 & 1E-19\\
    4 & 3.98E-02 & 2.20 & 1E-18 & 3.75E-01 & 1.32 & 1E-33\\
    5 & 2.57E-02 & 1.95 & 1E-36 & 2.90E-01 & 1.16 & 1E-33\\
    6 & 1.66E-02 & 2.41 & 1E-36 & 2.23E-01 & 1.44 & 1E-33\\
    7 & 1.21E-02 & 2.06 & 1E-36 & 1.84E-01 & 1.23 & 1E-33\\
    8 & 8.64E-03 & 2.50 & 1E-35 & 1.51E-01 & 1.50 & 1E-31\\
    9 & 6.73E-03 & 2.12 & 1E-32 & 1.30E-01 & 1.27 & 1E-28\\
    10 & 5.14E-03 & 2.55 & 1E-29 & 1.11E-01 & 1.53 & 1E-24\\
    11 & 4.19E-03 & 2.15 & 1E-27 & 9.77E-02 & 1.29 & 1E-22\\
    12 & 3.35E-03 & 2.59 & 1E-24 & 8.54E-02 & 1.55 & 1E-19\\
    13 & 2.81E-03 & 2.17 & 1E-21 & 7.70E-02 & 1.30 & 1E-17\\
    14 & 2.32E-03 & 2.61 & 1E-17 & 6.85E-02 & 1.57 & 1E-34\\
    15 & 1.99E-03 & 2.19 & 1E-38 & 6.26E-02 & 1.31 & 1E-35\\
    \hline
    \hline
    & \multicolumn{3}{ c|| }{2.19} & \multicolumn{3}{c|}{1.31}\\
    \hline
    \end{tabular}
    \label{tab:eig_p_1d}
    \caption{Convergence rate of the Fourier-Galerkin method for one dimensional
    Poisson problem. We list the magnitude of the error and the convergence
    exponent in both the $L^2$ and energy norms. The column $E$ is the estimate
    of the quadrature error in the computation of the error provided by SymPy's
    {\tt{quad}} function. The last row in the table has the estimate for the
    convergence exponent obtained by the least-squares fit.}
  \end{table}

  \subsection{One dimensional biharmonic problem}
  As in the previous section, in order to solve the one dimensional biharmonic
  problem
  \begin{equation}
  \label{eq:bih_strong_1}
  \begin{aligned}
    &\deriv{^4u}{x^4} = f\,\text{ in }(-1, 1),\\
    &u(-1) = u(1) = 0,\\
    &\deriv{^2u}{x^2}(-1) = \deriv{^2u}{x^2}(1) = 0
  \end{aligned}
  \end{equation}
  we shall first consider the corresponding eigenvalue problem
  \begin{equation}
  \begin{aligned}
    &\deriv{^4u}{x^4} = \lambda u,\text{ in }(-1, 1),\\
    &u(-1) = u(1) = 0,\\
    &\deriv{^2u}{x^2}(-1) = \deriv{^2u}{x^2}(1) = 0.
  \end{aligned}
  \end{equation}
  With the given boundary conditions, the operator acting on the left hand
  side is $\delta^2$. It thus has the same eigenfunctions as $\delta$ while
  the eigenvalues are squares of the eigenvalues of $\delta$, that is
  $\lambda_k=\alpha_k^4$.
  
  Our numerical method for solving (\ref{eq:bih_strong_1}) is based on the 
  weak formulation of the problem which reads: Find $u\in V\defeq\left\{v\in
  H^2((-1, 1)), u(x)=\deriv{^2u}{x^2}(x),\text{for}\,x=\pm 1\right\}$ such that
  \[
    \inner{\deriv{^2u}{x^2}}{\deriv{^2v}{x^2}} = \inner{f}{v},\,\forall v\in V.
  \]
  The space $V_n$ introduced in the previous section is the subspace of $V$
  and it is therefore valid to consider a problem: Find $u_n\in V_n$ such that
  \begin{equation}
    \label{eq:bih_weak_1}
    \inner{\deriv{^2u_n}{x^2}}{\deriv{^2v}{x^2}} = \inner{f}{v},\,\forall v\in V_n.
  \end{equation}
  We find that the corresponding linear system $\mathbb{A}\mathbf{u}=\mathbf{b}$
  is solved by vector $\mathbf{u}\in\R^n$ with components
  $\mathbf{u}_i=\frac{\mathbf{b}_i}{\alpha_i^4}$ and consequently the error of the
  method can be expressed as
  \[
    e_n = \sum_{k=n}^{\infty} \frac{\inner{f}{\phi_K}}{\alpha_k^4} \phi_k.
  \]

  Convergence of the method in the $L^2$ norm follows from the estimate
  \[
    \norm{e_n}^2=\inner{e_n}{e_n} = \sum_{k=n}^{\infty}\frac{\inner{f}{\phi_k}^2}{\alpha^8_k} \leq
      \frac{1}{\alpha^8_n}\sum_{k=n}^{\infty} (f, \phi_k)^2 \leq
      \frac{\norm{f}^2}{\alpha^8_n}
  \]
  which yields $\norm{e}\leq\frac{2^4}{\left(\pi + n\pi\right)^4}\norm{f}$.
  We have $\norm{e}$ decreasing as $\mathcal{O}(n^{-4})$. Since each
  differentiation of the eigenfunction  $\phi_k$ produces a factor $\alpha_k$
  we get the estimates of the error in 
  $H^1$ and $H^2$ seminorms as
  \begin{align*}
    \seminorm{e_n}_1 &\leq \frac{1}{\alpha^3_n}\norm{f} = \frac{2^3}{\left(\pi + n\pi\right)^3}, \\
    \seminorm{e_n}_2 &\leq \frac{1}{\alpha^2_n}\norm{f} = \frac{2^2}{\left(\pi + n\pi\right)^2}.
  \end{align*}
  The errors in the respected seminorms should therefore behave as
  $n^{-3}$ and $n^{-2}$.
    
  Table \ref{tab:eig_b_1d} shows results from the convergence study we run to
  verify convergence properties of the method established in the previous
  paragraphs. In this study, the exact solution of (\ref{eq:bih_strong_1}) was
  computed to match the right hand side $f=x e^x$. The results show very good
  agreement with the theoretical estimates.

  \begin{table}
    \centering
    \begin{tabular}{|c|c|c|c||c|c|c||c|c|c|}
    \hline
    $n$ & $\norm{e}$ & $p$ & $E$ & $\seminorm{e}_1$ & $p$ & $E$ & $\seminorm{e}_2$ & $p$ & $E$\\
\hline
3 & 3.08E-04 & 2.68 & 1E-27 & 2.00E-03 & 2.01 & 1E-25 & 1.33E-02 & 1.37 & 1E-25\\
4 & 8.63E-05 & 4.42 & 1E-21 & 7.28E-04 & 3.50 & 1E-19 & 6.41E-03 & 2.54 & 1E-21\\
5 & 4.42E-05 & 3.00 & 1E-20 & 4.38E-04 & 2.28 & 1E-19 & 4.49E-03 & 1.60 & 1E-18\\
6 & 1.81E-05 & 4.90 & 1E-34 & 2.16E-04 & 3.86 & 1E-41 & 2.70E-03 & 2.79 & 1E-38\\
7 & 1.12E-05 & 3.10 & 1E-35 & 1.50E-04 & 2.38 & 1E-41 & 2.09E-03 & 1.68 & 1E-39\\
8 & 5.65E-06 & 5.14 & 1E-37 & 8.76E-05 & 4.03 & 1E-41 & 1.42E-03 & 2.90 & 1E-38\\
9 & 3.90E-06 & 3.15 & 1E-36 & 6.58E-05 & 2.42 & 1E-39 & 1.16E-03 & 1.72 & 1E-36\\
10 & 2.24E-06 & 5.27 & 1E-17 & 4.26E-05 & 4.13 & 1E-33 & 8.47E-04 & 2.96 & 1E-32\\
11 & 1.65E-06 & 3.17 & 1E-33 & 3.37E-05 & 2.45 & 1E-32 & 7.17E-04 & 1.74 & 1E-30\\
12 & 1.04E-06 & 5.36 & 1E-28 & 2.34E-05 & 4.19 & 1E-26 & 5.53E-04 & 3.00 & 1E-25\\
13 & 8.03E-07 & 3.19 & 1E-26 & 1.92E-05 & 2.47 & 1E-24 & 4.80E-04 & 1.76 & 1E-22\\
14 & 5.38E-07 & 5.42 & 1E-22 & 1.41E-05 & 4.23 & 1E-20 & 3.84E-04 & 3.03 & 1E-18\\
15 & 4.31E-07 & 3.21 & 1E-21 & 1.18E-05 & 2.48 & 1E-19 & 3.40E-04 & 1.77 & 1E-17\\
\hline
    \hline
    & \multicolumn{3}{c||}{3.93} &
      \multicolumn{3}{c||}{3.05} &
      \multicolumn{3}{c|}{2.18}\\
    \hline
    \end{tabular}
    \label{tab:eig_b_1d}
    \caption{Convergence rate of the Fourier-Galerkin method for one dimensional
    biharmonic problem. We list the magnitude of the error and the convergence
    exponent in the $L^2$, and $H^2$ norms as well as $H^1$ seminorm. The column
    $E$ is the estimate of the quadrature error in the computation of the error
    provided by SymPy's {\tt{quad}} function. The last row in the table has the
    estimate for the convergence exponent obtained by the least-squares fit.}
  \end{table}

  We have demonstrated by both theoretical estimates and the convergence studies
  that the presented Fourier-Galerkin method can be used to solve accurately 
  the one dimensional Poisson and biharmonic problems. We therefore have at our
  disposal a numerical method to resolve the beam physics of cases (i) and (ii).
  In the next sections we show how the Fourier basis can be used to construct
  a numerical method for solving Poisson and biharmonic problems in two dimensional
  domain, that is to resolve the physical processes on the beam considered
  in this note.

  \subsection{Two dimensional Poisson problem}
  We are interested in finding a solution to the Poisson problem on a bi-unit
  square with homogeneous Dirichlet boundary conditions on the boundary of the
  square
  \begin{equation}
  \label{eq:poisson_strong_2}
  \begin{aligned}
    -\Delta u &= f\,\text{ in }\mathcal{P},\\
            u &= 0\,\text{ on }\partial\mathcal{P}.
  \end{aligned}
  \end{equation}
  A variational formulation of (\ref{eq:poisson_strong_2}) reads: Find $u\in
  V\defeq H^1_0\left(\mathcal{P}\right)$ such that for all $v\in V$ it holds
  that
  \begin{equation}
    \label{eq:poisson_weak_2}
    \Inner{\Grad{u}}{\Grad{v}} = \Inner{f}{v},
  \end{equation}
  where $\Inner{\cdot}{\cdot}$ denotes the $L^2$ inner product over $\mathcal{P}$.
  We define $V_n$, a finite dimensional space in which the approximation of $u$
  is sought as $V_n=\spn\left\{\phi_i(x)\phi_j(y)\right\}_{i, j = 0, 0}^{n-1, n-1}$.
  Clearly, $V_n\subset V$ and $\text{dim}V_n = n^2$. Moreover, each basis function of
  $\phi_{i, j}, \phi_{i, j}(x, y)\defeq\phi_i(x)\phi_j(y)$, is an eigenfunction
  of the Laplace operator acting on functions in $\mathcal{P}$ with zero boundary
  values. We have that $\lambda_{i, j}=\lambda_i + \lambda_j$, where $\lambda_i$
  is the $i$-th eigenvalue of $\delta$ is an eigenvalue for function $\phi_{i, j}$.
  Finally we note that the basis functions of $V_n$ are orthonormal in the
  $\Inner{\cdot}{\cdot}$ inner product. Considering the problem (\ref{eq:poisson_strong_2})
  we get that the expansion coefficients $\mathbb{U}\in\R^{n\times n}$ of the
  approximation of $u$ in $V_n$
  \[
    u_n=\sum\limits_{i, j = 0, 0}^{n-1, n-1}\mathbb{U}_{i,
    j}\phi_{i, j}
  \]
  are given as a solution of the linear system
  \begin{equation}
    \label{eq:p_sys}
    \Lambda\mathbb{U} + \mathbb{U}\Lambda = \mathbb{B},
  \end{equation}
  where the matrix $\mathbb{B}\in\R^{n\times n}$ has $\mathbb{B}_{i,
  j}=\Inner{v}{\phi_{i, j}}$. Further, matrix $\Lambda\in\R^{n\times n}$ is
  a diagonal matrix with diagonal elements $\Lambda_{i, i}$ equal to the
  eigenvalues $\lambda_i$.
  
  At this point we take a step back and compare (\ref{eq:p_sys}) to the systems
  obtained if (\ref{eq:poisson_weak_2}) were considered on subspace $V_n$
  constructed from functions other than eigenfunction of the Laplacian, e.g.
  polynomials. In such case, the expansion coefficients of the approximate
  solution are given as a solution to 
  \begin{equation}
    \label{eq:p_tensor}
    \tilde{\mathbb{A}}\mathbb{U}\tilde{\mathbb{M}} +
    \tilde{\mathbb{M}}\mathbb{U}\tilde{\mathbb{A}} =
    \tilde{\mathbb{B}},
  \end{equation} 
  with $\tilde{\mathbb{A}}$ and $\tilde{\mathbb{M}}$ the matrices of one
  dimensional Laplace and identity operators with respect to the chosen basis.
  We note that given an invertible map 
  $\rho:\left[0, n^2-1\right]\mapsto\left[0, n-1\right]\times\left[0, n-1\right]$
  (e.g. collapsing by row), the system (\ref{eq:p_tensor}) can be written as
  \[
  \mathcal{A}\mathbf{u} = \mathbf{b}.
  \]
  Here $\mathbf{u}_{k}=\mathbb{U}_{\rho(k)}, \mathbf{b}_{k}=\mathbb{B}_{\rho(k)}$
  and $\mathcal{A}\in\R^{n^2\times n^2}, \mathcal{A}=\mathbb{A}\otimes\mathbb{M}
  +\mathbb{M}\otimes\mathbb{A}$. Clearly this system requires more storage than
  (\ref{eq:p_tensor}). Moreover, (\ref{eq:p_tensor}) can be solved efficiently
  by the fast tensor product method using transformation matrices $\mathbb{Q}$
  with columns from vectors solving a generalized matrix eigenvalue problem
  $\tilde{\mathbb{A}}q=\tilde{\lambda}\tilde{\mathbb{M}}q$. Setting 
  $\mathbb{U}^\prime = \mathbb{Q}\mathbb{U}\mathbb{Q}^{\text{T}}$ and using
  $\mathbb{Q}^{\text{T}}\tilde{\mathbb{M}}\mathbb{Q}=\mathbb{I}$ together with
  $\mathbb{Q}^{\text{T}}\tilde{\mathbb{A}}\mathbb{Q}=\tilde{\Lambda}$ we
  transform the system into eigenspace
  \[
    \tilde{\Lambda}\mathbb{U}^{\prime} + \mathbb{U}^{\prime}\tilde{\Lambda} = 
    \mathbb{Q}^{\text{T}}\tilde{\mathbb{B}}\mathbb{Q}.
  \]
  In this setting, the choice of the Fourier basis presents a major advantage
  since we already work in an eigenspace and thus have $\mathbb{Q}=\mathbb{I}$
  while $\Lambda = \tilde{\Lambda}$. Therefore, there is no need for the solution
  of the generalized eigenvalue problem and we can immediately write the
  solution of (\ref{eq:p_sys})
  \[
    \mathbb{U}_{i, j} = \frac{\mathbb{B}_{i, j}}{\lambda_i + \lambda_j}.
  \]
  Consequently, the approximation $u_n$ takes the form
  \[
    u_n = \sum\limits_{i, j = 0, 0}^{n-1, n-1}\frac{\Inner{f}{\phi_{i, j}}}{\lambda_i + \lambda_j}
    \phi_{i, j}
  \]
  which should be contrasted with the exact solution
  \[
    u = \sum\limits_{i, j = 0, 0}^{\infty, \infty}\frac{\Inner{f}{\phi_{i, j}}}{\lambda_i + \lambda_j}
    \phi_{i, j}.
  \]
  As in the one dimensional case we see that the numerical solution due to
  Fourier-Galerkin method is $u_n = -\Delta|_{n}^{-1}f$ where
  $-\Delta|_{n}^{-1}=P_n \circ \left(-\Delta^{-1}\right)$ is a projection by 
  the operator $P_n, P_n:V\mapsto V_n$, of the inverse of the Laplacian on the
  subspace $V_n$.

  The convergence rate of the method in the $L^2$ and the energy norm follow
  from the estimates which are obtained using same ideas as in the one
  dimensional case
  \[
    \norm{e}^2 = \sum\limits_{i, j = n, n}^{\infty,
    \infty}\frac{\Inner{f}{\phi_{i, j}}^2}{\left(\lambda_i + \lambda_j\right)^2}
      \leq
      \frac{\norm{f}^2}{\left(2\lambda_n\right)^2}
  \]
  and
\[
    \norm{e}_1^2 = \sum\limits_{i, j = n, n}^{\infty,
    \infty}\frac{\Inner{f}{\phi_{i, j}}^2}{\left(\lambda_i + \lambda_j\right)}
      \leq
      \frac{\norm{f}^2}{2\lambda_n}
  \]
  We thus have $\norm{e}\leq\frac{2\norm{f}}{\left(\pi + n\pi\right)^2}$ and
  $\norm{e}_1\leq\frac{\sqrt{2}\norm{f}}{\pi + n\pi}$.

  The estimates were tested by a convergence study with the exact solution
  of \ref{eq:poisson_strong_2} given as $u=(x^2 - 1)e^x(y^2 - 1)$. We note that
  as for the one dimensional Poisson problem theoretical estimates seem to
  slightly overestimate the numerical rates.

  \begin{table}
    \centering
    \begin{tabular}{|c|c|c|c||c|c|c|}
    \hline
    $n$ & $\norm{e}$ & $p$ & $E$ & $\norm{e}_1$ & $p$ & $E$\\
      \hline
      3 & 7.82E-02 & 1.75 & 1E-21 & 5.87E-01 & 1.07 & 1E-19\\
      4 & 4.24E-02 & 2.12 & 1E-18 & 4.04E-01 & 1.30 & 1E-33\\
      5 & 2.69E-02 & 2.05 & 1E-36 & 3.07E-01 & 1.24 & 1E-34\\
      6 & 1.76E-02 & 2.32 & 1E-36 & 2.37E-01 & 1.40 & 1E-33\\
      7 & 1.26E-02 & 2.16 & 1E-36 & 1.94E-01 & 1.30 & 1E-33\\
      8 & 9.16E-03 & 2.41 & 1E-35 & 1.60E-01 & 1.45 & 1E-31\\
      9 & 7.05E-03 & 2.21 & 1E-32 & 1.37E-01 & 1.33 & 1E-28\\
      10 & 5.45E-03 & 2.46 & 1E-29 & 1.17E-01 & 1.48 & 1E-24\\
      11 & 4.39E-03 & 2.25 & 1E-27 & 1.03E-01 & 1.35 & 1E-22\\
      12 & 3.54E-03 & 2.49 & 1E-24 & 9.04E-02 & 1.50 & 1E-19\\
      13 & 2.95E-03 & 2.27 & 1E-21 & 8.10E-02 & 1.36 & 1E-17\\
      14 & 2.45E-03 & 2.51 & 1E-17 & 7.24E-02 & 1.51 & 1E-35\\
      15 & 2.09E-03 & 2.29 & 1E-39 & 6.59E-02 & 1.37 & 1E-35\\
      \hline
    \hline
    \hline
    & \multicolumn{3}{ c|| }{2.19} & \multicolumn{3}{c|}{1.33}\\
    \hline
    \end{tabular}
    \label{tab:eig_p_2d}
    \caption{Convergence rate of the Fourier-Galerkin method for two dimensional
    Poisson problem. We list the magnitude of the error and the convergence
    exponent in both the $L^2$ and energy norms. The column $E$ is the estimate
    of the quadrature error in the computation of the error provided by SymPy's
    {\tt{quad}} function. The last row in the table has the estimate for the
    convergence exponent obtained by the least-squares fit.}
  \end{table}

  \subsection{Two dimensional biharmonic problem}
  Finally, we would like to find a solution to the two dimensional biharmonic
  problem
  \begin{equation}
  \label{eq:bih_strong_2}
  \begin{aligned}
    \Delta^2 u &= f\,\text{ in }\mathcal{P},\\
            u &= 0\,\text{ on }\partial\mathcal{P},\\
    \Delta u &= 0\,\text{ on }\partial\mathcal{P}.
  \end{aligned}
  \end{equation}
  whose weak formulation reads: Find $u\in V\defeq\left\{v\in
    H^2\left(\mathcal{P}\right), v=\Delta{v}=0\text{ on
}\partial\mathcal{P}\right\}$ such that 
  \begin{equation}
  \label{eq:bih_weak_2}
  \Inner{\Delta u}{\Delta v} = \Inner{f}{v}\,\forall v \in V.
  \end{equation}
  Since the biharmonic operator in (\ref{eq:bih_strong_2}) is a square of the
  Laplace operator in (\ref{eq:poisson_strong_2}), we observe that $\phi_{i, j}$
  is also a solution of
  \[
  \begin{aligned}
    \Delta^2 u &= \lambda u \,\text{ in }\mathcal{P},\\
            u &= 0\,\text{ on }\partial\mathcal{P},\\
    \Delta u &= 0\,\text{ on }\partial\mathcal{P}
  \end{aligned}
  \]
  with the eigenvalue $\lambda_{i, j} = \alpha_i^4 + 2\alpha_i\alpha_j +
  \alpha_j^4$, i.e. the eigenvalues are simply the squares of the
  eigenvalues of the Laplace operator.

    Setting $V_n=\spn\left\{\phi_{i, j}\right\}_{i, j = 0, 0}^{n-1, n-1}$ we have
  that the expansion coefficients of the approximate solution $u_n$ of (\ref{eq:bih_weak_2})
  on $V_n$ are obtained from the system
  \begin{equation}
    \label{eq:b_sys}
    \Lambda^2\mathbb{U} + 2\Lambda\mathbb{U}\Lambda + \mathbb{U}\Lambda^2 = \mathbb{B},
  \end{equation}
  where the diagonal matrix $\Lambda$ has $\Lambda_{i, i}=\alpha_i^2$. The
  solution to the system is easily found to be
  \[
    \mathbb{U}_{i, j} = \frac{\mathbb{B}_{i, j}}{\alpha_i^4 +
    2\alpha_i^2\alpha_j^2 + \alpha_j^4}.
  \]
  As in the case of two dimensional Laplacian, we shall compare system
  (\ref{eq:b_sys}) to the system obtained with the Galerkin method and a
  subspace $V_n$ constructed from functions which in general are not the
  eigenfunction of the biharmonic operator. The expansion coefficients of
  the solution $u_n$ are then coming from the system
  \[
    \tilde{\mathbb{C}}\mathbb{U}\tilde{\mathbb{M}} +
    \tilde{\mathbb{A}}\mathbb{U}\tilde{\mathbb{A}} +
    \tilde{\mathbb{M}}\mathbb{U}\tilde{\mathbb{C}} =
    \tilde{\mathbb{B}},
  \]
  where we have reused $\tilde{\mathbb{M}}, \tilde{\mathbb{A}}$ to denote
  respectively the matrix of the one-dimensional identity and Laplace operators
  with respect to the basis, while $\tilde{\mathbb{C}}$ is the matrix of the one
  dimension biharmonic operator. To solve the system by a fast tensor product
  method, we first find the eigenvalues and eigenvectors of the problem 
  $\tilde{\mathbb{C}}q=\tilde{\lambda}^2\tilde{\mathbb{M}}q$ and transform the
  system into 
  \begin{equation}
    \label{eq:b_sys_2}
    \tilde{\Lambda}^2\mathbb{U}^{\prime} +
    \left(\mathbb{Q}^{\text{T}}\tilde{\mathbb{A}}\mathbb{Q}\right)\mathbb{U}^{\prime}
\left(\mathbb{Q}^{\text{T}}\tilde{\mathbb{A}}\mathbb{Q}\right) + 
    \mathbb{U}^{\prime}\tilde{\Lambda}^2 =
    \mathbb{Q}^T\tilde{\mathbb{B}}\mathbb{Q}.
  \end{equation} 
  The cost of the solution of (\ref{eq:b_sys_2}) then hinges on whether
  $\mathbb{Q}$ are also the eigenvectors of
  $\tilde{\mathbb{A}}q=\beta\tilde{\mathbb{M}}q$.
  If matrices $\tilde{\mathbb{A}}, \tilde{\mathbb{C}}$ share the eigenvectors,
  then the system is cheap to solve. However this is typically not the case.
  With $V_n$ constructed from eigenfunctions $\phi_{i, j}$ we get 
  $\tilde{\mathbb{A}}\mathbb{Q}=\tilde{\mathbb{M}}\mathbb{Q}\tilde{\Lambda}$ while
  $\tilde{\mathbb{C}}\mathbb{Q}=\tilde{\mathbb{M}}\mathbb{Q}\tilde{\Lambda}^2$
  and plugging into (\ref{eq:b_sys_2}) we see that the system is in fact the
  assembled system (\ref{eq:b_sys}).

  Convergence estimates for the method are based on the expression for the error
  $e_n=u-u_n$ which reads
  \[
    e_n = \sum\limits_{i, j = n, n}^{\infty, \infty} \frac{\Inner{\phi_{i,
    j}}{f}}{\alpha_i^4 + 2\alpha_i^2\alpha_j^2 + \alpha_j^4}\phi_{i, j}.
  \]
  Using ideas from the previous sections we get the upper estimate for the error
  in $L^2$ norm
  as
  \[
    \norm{e} \leq \frac{4\norm{f}}{\left(\pi + n\pi \right)^4}.
  \]
  The error estimates in $H^1$ and $H^2$ seminorms are then given as 
  as
  \[
    \seminorm{e}_1 \leq \frac{2\sqrt{2}\norm{f}}{\left(\pi + n\pi \right)^3}
  \]
  and
  \[
    \seminorm{e}_2 \leq \frac{2\norm{f}}{\left(\pi + n\pi \right)^2}.
  \]
  % Finally beharmonic operator section should follow same straucture
  % results as plot
  % We have method for physics of plate problems.
  % The properties are not due to sines cosis but beacuse we have the
  % eigenvectors!
  \subsection{Galerkin method with the basis of Legendre polynomials for the
  Poisson problem}
  % Shoortly introduce Shen's method and verify our implementation for 1d adn
  % 2d problem. 
  % Condition numbers of A, M in both cases
  % Can this be tied to rate?
  % Why don't we see scaling of the differential operator in the Shen case
  % results as plot

  % If you could come up with u, u`` basis with Leg, it could go here and
  % we'd again show plots with rates.
 
\end{document}
